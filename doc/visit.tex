\Damaris{} can be used to perform in situ visualization with VisIt through the \emph{libsimV2} interface. 
VisIt must be installed on your platform (see the VisIt website at \url{https://wci.llnl.gov/codes/visit/}).
Follow the installation process of \Damaris{} presented in Chapter~\ref{chap:downloadingAndInstalling}
and enable VisIt by providing the VISIT\_ROOT 
VisIt install directory in the CMakeLists.txt (see Table~\ref{tab:cmake}).
\textbf{Warning:} \emph{libsimV2} seems to have some problems on some platforms. We successfully
tested \Damaris{} with VisIt 2.5.2 and 2.8.0. Versions earlier than 2.5.2 will not work with \Damaris{},
and du to various bugs in \emph{libsimV2}, we haven't managed to have \Damaris{} use any of the
versions between 2.5.2 and 2.8.0. Version 2.8.0 seems to still suffer from problems on some platforms.

%==============================================================================%
%==============================================================================%
\section{VisIt options}

VisIt has to be enabled through the configuration file as shown in
Listing~\ref{VisItXML}. The \texttt{<path>} tag locates the installation
directory of VisIt, it is used by the simulation to load VisIt plugins.
The \texttt{<options>} tag provides an equivalent to VisIt's command line options (see
the VisIt manual). Most of the time, the \texttt{<options>} section is not required.

\noindent\begin{minipage}{\textwidth}
\vspace{0.5cm}
\lstset{language=XML,caption={Providing VisIt options},label=VisItXML}
\lstinputlisting[language=XML]{listings/visit.xml}
\end{minipage}

%==============================================================================%
%==============================================================================%
\section{Description of meshes, curves and fields}

\subsection{Meshes}

The current version of \Damaris{} supports the following meshes:
rectilinear meshes, curvilinear meshes, and point meshes. This section shows how to expose
a mesh and how to map a variable onto a mesh.
Just like other \Damaris{} objects, meshes have to be described in the XML file, as shown
in Listing~\ref{MeshVisItXML}. This kind of description should be placed in the \texttt{<data>} section
of the configuration, at the root or within any subgroup.

\noindent\begin{minipage}{\textwidth}
\vspace{0.5cm}
\lstset{language=XML,caption={Describing a Mesh in \Damaris{}},label=MeshVisItXML}
\lstinputlisting[language=XML]{listings/mesh.xml}
\end{minipage}

A mesh is characterized by a \texttt{name}, a \texttt{type} (currently only ``rectilinear'', ``curvilinear'' or
``point'' are allowed), and a topological dimension (for instance a surface in a 3D space is characterized by three
coordinates for each of its points, but has a 2D topology).

Finally two to three coordinates have to be provided through the \texttt{<coord>} tag. 
These coordinates refer to existing variables from the configuration.
The name of these variables can be absolute or relative, as it is search the same way a variable searches
its layout. The \texttt{unit} of a coordinate, when provided, hides the \texttt{unit} of the corresponding variable.
The \texttt{label} is used when drawing figures.

\subsection{Fields}

A field is a value that can be placed on each node or each zone of a mesh.
To allow a described variable to be mapped onto a mesh, some new attributes have to be added in its description.
Listing~\ref{xml:field} shows the description of a variable with attributes related to visualization.

\noindent\begin{minipage}{\textwidth}
\vspace{0.5cm}
\lstset{language=XML,caption={Extended variable description to expose it as a field},label=xml:field}
\lstinputlisting[language=XML]{listings/field.xml}
\end{minipage}

The most important attribute is \texttt{mesh}, which is a reference to an existing mesh on which
to draw the field. This reference should be an absolute name (\Damaris{} is not yet capable of searching
for a local mesh within the group where the variable is defined). 
The \texttt{centering} attribute can be either ``nodal'' (values correspond to vertices 
of the mesh) or ``zonal'' (values correspond to zones in the mesh). The \texttt{type} should be
``scalar'' (other types are not yet supported by \Damaris).

By default, \texttt{visualizable} and \texttt{time-varying} are already set to ``true'', \texttt{unit} can be omitted.
The other attributes (\texttt{type} and \texttt{centering}) can be omitted if the variable is not concerned by
visualization.
The ``rule'' is that variables that serve as coordinates in meshes are not visualizable, while
other variable are considered as fields and are visualizable, thus must have an attached mesh.

\subsection{Curves}

\Damaris{} can expose curves the same way it exposes meshes. Listing~\ref{CurveVisItXML}
shows how to expose a curve. A curve has exactly two coordinates (the $x$ and $y$ coordinates
of the points of the curve) refering to variables the same way meshes use coordinate variables.

An important difference between meshes and curves is that a curve is only produced by
the client 0 of the simulation. That is, only this client has to write the coordinate variables.
If other clients write coordinate variables, they will be discarded by VisIt.

\noindent\begin{minipage}{\textwidth}
\vspace{0.5cm}
\lstset{language=XML,caption={Describing a Mesh in \Damaris{}},label=CurveVisItXML}
\lstinputlisting[language=XML]{listings/curve.xml}
\end{minipage}

%==============================================================================%
%==============================================================================%
\section{External actions and interactivity}

Events can be set as ``external'' by adding the \texttt{external="true"} attribute to their
description. An external event can be fired from the VisIt client itself. These events are exposed as commands
and are accessible through the VisIt viewer's commands interface.
Clicking on a command will trigger the corresponding event in all dedicated processes (or all processes
in synchronous mode). It is ensured that all the actions will be triggered at the same time, thus
the actions can safely use global communications using MPI.
Triggering an event from VisIt is equivalent to sending a ``bcast''-type event from a client.