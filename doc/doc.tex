\documentclass[11pt]{report} 
\usepackage{url}
\usepackage{anysize}
\usepackage{makeidx}
\usepackage{hyperref}
\usepackage{listings}
\usepackage{color}

\marginsize{2cm}{2cm}{2cm}{2cm}
 
\definecolor{dkgreen}{rgb}{0,0.6,0}
\definecolor{mauve}{rgb}{0.58,0,0.82}
\definecolor{gray}{rgb}{0.4,0.4,0.4}
\definecolor{darkblue}{rgb}{0.0,0.0,0.6}
\definecolor{cyan}{rgb}{0.0,0.0,1}
 
\lstset{ %
  language=C,                % the language of the code
  basicstyle=\footnotesize,           % the size of the fonts that are used for the code
  %numbers=left,                   % where to put the line-numbers
  %numberstyle=\tiny\color{black},  % the style that is used for the line-numbers
  %stepnumber=1,                   % the step between two line-numbers. If it's 1, each line 
                                  % will be numbered
  numbersep=5pt,                  % how far the line-numbers are from the code
  backgroundcolor=\color{white},      % choose the background color. You must add \usepackage{color}
  showspaces=false,               % show spaces adding particular underscores
  showstringspaces=false,         % underline spaces within strings
  showtabs=false,                 % show tabs within strings adding particular underscores
  frame=single,                   % adds a frame around the code
  frameround=ftft,
  float=th,
  rulecolor=\color{black},        % if not set, the frame-color may be changed on line-breaks within not-black text (e.g. commens (green here))
  tabsize=2,                      % sets default tabsize to 2 spaces
  captionpos=b,                   % sets the caption-position to bottom
  breaklines=true,                % sets automatic line breaking
  breakatwhitespace=false,        % sets if automatic breaks should only happen at whitespace
  title=\lstname,                   % show the filename of files included with \lstinputlisting;
                                  % also try caption instead of title
  keywordstyle=\color{blue},          % keyword style
  commentstyle=\color{dkgreen},       % comment style
  stringstyle=\color{mauve},         % string literal style
  escapeinside={\%*}{*)},            % if you want to add a comment within your code
  morekeywords={*,...}               % if you want to add more keywords to the set
}

\lstdefinelanguage{XML}
{
  morestring=[b]",
  %morestring=[s]{>}{<},
  morecomment=[s]{<?}{?>},
  morecomment=[s]{<!--}{-->},
  stringstyle=\color{mauve},
  identifierstyle=\color{blue},
  keywordstyle=\color{blue},
  morekeywords={xmlns,version,type,simulation,architecture,
  cores,clients,buffer,queue,data,actions,parameter,layout,variable,
  group}% list your attributes here
}

\newcommand{\Damaris}{\emph{\textbf{Damaris}}}
\newcommand{\installdir}[1]{\texttt{\$HOME/local#1}}
\newcommand{\file}[1]{\emph{#1}}
\newcommand{\function}[1]{\texttt{#1}}

\makeindex

%%%%%%%%%%%%%%%%%%%%%%%%%%%%%%%%%%%%%%%%%%%%%%%%%%%%
%%%%%%%%%%%%%%%%%%%%%%%%%%%%%%%%%%%%%%%%%%%%%%%%%%%%
%%%%%%%%%%%%%%%%%%%%%%%%%%%%%%%%%%%%%%%%%%%%%%%%%%%%
\begin{document}

\title{\Huge{ 
	 Getting Started with \Damaris{}} \\
	\normalsize{} version 0.7}

\author{\textbf{Matthieu Dorier}}
\date{December 2012}
\maketitle

\setcounter{tocdepth}{1}
\tableofcontents

%%%%%%%%%%%%%%%%%%%%%%%%%%%%%%%%%%%%%%%%%%%%%%%%%%%%
%%%%%%%%%%%%%%%%%%%%%%%%%%%%%%%%%%%%%%%%%%%%%%%%%%%%
%%%%%%%%%%%%%%%%%%%%%%%%%%%%%%%%%%%%%%%%%%%%%%%%%%%%
%%%%%%%%%%%%%%%%%%%%%%%%%%%%%%%%%%%%%%%%%%%%%%%%%%%%
%%%%%%%%%%%%%%%%%%%%%%%%%%%%%%%%%%%%%%%%%%%%%%%%%%%%
%%%%%%%%%%%%%%%%%%%%%%%%%%%%%%%%%%%%%%%%%%%%%%%%%%%%
\chapter*{The \Damaris{} approach}

Large-scale simulations usually perform I/O in a time-partitioning manner.
The simulation periodically stops and all processes perform a write (for checkpointing
or future analysis purpose) concurrently to a parallel file system. This approach has also been
used to perform in-situ visualization, where the simulation periodically stops to compute images.

On recent infrastructures, the large number of cores available in each SMP node makes
it possible to offload these tasks in order to gain efficiency by working in a space-partitioning manner.
\Damaris{} proposes to dedicate one core on each SMP node to perform these tasks.
It uses shared memory to communicate data from the cores running the simulation's code and
the cores dedicated to these set-aside, data intensive tasks.

A plugin system is used to enrich \Damaris{} with new functionalities. Plugins can be written
in C++ or Python, making asynchronous, in-situ data management straightforward for end-users.
Besides, \Damaris{} can be connected to the VisIt visualization software to provide
non-impacting, in-situ, interactive visualization to existing simulations.



%%%%%%%%%%%%%%%%%%%%%%%%%%%%%%%%%%%%%%%%%%%%%%%%%%%%
%%%%%%%%%%%%%%%%%%%%%%%%%%%%%%%%%%%%%%%%%%%%%%%%%%%%
%%%%%%%%%%%%%%%%%%%%%%%%%%%%%%%%%%%%%%%%%%%%%%%%%%%%
%%%%%%%%%%%%%%%%%%%%%%%%%%%%%%%%%%%%%%%%%%%%%%%%%%%%
%%%%%%%%%%%%%%%%%%%%%%%%%%%%%%%%%%%%%%%%%%%%%%%%%%%%
%%%%%%%%%%%%%%%%%%%%%%%%%%%%%%%%%%%%%%%%%%%%%%%%%%%%
\chapter{Downloading and installing}\label{chap:downloadingAndInstalling}

%==============================================================================%
%==============================================================================%
\section{Dependencies}\label{sec:dependencies}

In the following, we will assume that you install all dependencies locally in \installdir.
\Damaris{} requires the following dependencies.
\begin{itemize}
	\item CMake (2.8 or newer), \\ available at \url{http://www.cmake.org/}
	\item The XSD code synthesis library, \\ available at \url{http://www.codesynthesis.com/products/xsd/}
	\item The Xerces-C library (version 3.1.1 or newer), \\
	available at \url{http://xerces.apache.org/xerces-c/download.cgi}
	\item The following parts of the Boost C++ library (version 1.47 or newer): python (if you enable the 
	Python support), filesystem, system and program\_options, available at \url{http://www.boost.org/}
\end{itemize}

Additionally, the Python support and the VisIt support can be enabled by installing the corresponding
libraries. See Chapters~\ref{Python} and~\ref{chap:VisIt} for this (in the right order: start by installing VisIt, then
Python, then come back to this main installation process).

\subsection{XSD codesynthesis}

Download the XSD sources for your platform.
Untar/unzip it and install it by copying it into the \installdir directory. 
(Only the headers are required, you should copy the \texttt{cxx} folder into \installdir{/include}).

\subsection{Xerces-C}

Download Xerces-C (version 3.1.1 or greater). Take the source for your platform.
Configure and compile:

\begin{verbatim}
./configure --prefix=$HOME/local --disable-threads --disable-network
make
make install
\end{verbatim}

\subsection{Boost}

Download Boost (version 1.51 or higher). Configure and compile it:

\begin{verbatim}
./bootstrap.sh --prefix=$HOME/local --with-python-root=$HOME/local \
       --with-libraries=date_time,program_options,system,filesystem,python 
./b2
./b2 install
\end{verbatim}

\noindent\textbf{Note:} delete ``python'' and ``--with-python-root=\$HOME/local'' if you do not intend 
to use the Python support in \Damaris{}. If you use the Python interface for \Damaris{}, check
Chapter~\ref{Python}.\newline

\noindent\textbf{Note:} install VisIt first if you intend to use the VisIt support. VisIt also depends on Python
so if you use VisIt and Python, make sure the same Python installation is used both for
VisIt and Damaris. See Chapter~\ref{chap:VisIt} for more information on installing VisIt.\newline

\noindent\textbf{Note:} \Damaris{} uses MPI\_Get\_processor\_name and expect it to return the same
name on all the cores of the same SMP node. It may not be the case on some platform. On BlueGene/P, don't forget to use the BGP flag (see the CMakeLists.txt file) to use BGP's personality
functions instead.

%==============================================================================%
%==============================================================================%
\section{Compiling \Damaris{}}

Download and untar the \Damaris{} archive. \Damaris{} uses CMake to be compiled.
Make sure the \texttt{cmake} command is in your PATH environment variable.

In the \Damaris{} root directory, the \file{CMakeLists.txt} file contains the compilation directives
and various options. Open this file and modify the lines listed in Table~\ref{tab:cmake}.
When these modifications are done, type
\begin{verbatim}
cmake -G "Unix Makefiles"
\end{verbatim}
This will generate the \file{Makefile} to compile \Damaris{}. Go on with 
\begin{verbatim}
make
\end{verbatim}
You're done!

\begin{table}
\centering
\begin{tabular}{|l|l|l|}
	\hline
   Line & Variable & Content \\
   \hline
   \hline
   13 &  & Name of the C++ MPI compiler. \\
   14 &  & Name of the C MPI compiler. \\
   15 &  & Name of the Fortran MPI compiler (uncomment if needed). \\
   27 & EXTERNAL\_ROOT & Prefix path where dependencies have been installed. \\
   47 & BGP & Uncomment if you run on BlueGene/P. \\
   67 & BOOST\_ROOT & Boost library location. \\
   77 & XSD\_ROOT & XSD library location. \\
   84 & XERCESC\_ROOT & Xerces-C library location. \\
   94 & PYTHON\_ROOT & Python library location. \\
   95 & NUMPY\_INCLUDE\_DIR & Location of the NumPy headers. \\
   134 & VISIT\_ROOT & Location of VisIt. Comment if you don't need VisIt support. \\
   \hline
\end{tabular}\caption{CMakeLists.txt modifications before compiling \Damaris{}}\label{tab:cmake}
\end{table}



%%%%%%%%%%%%%%%%%%%%%%%%%%%%%%%%%%%%%%%%%%%%%%%%%%%%
%%%%%%%%%%%%%%%%%%%%%%%%%%%%%%%%%%%%%%%%%%%%%%%%%%%%
%%%%%%%%%%%%%%%%%%%%%%%%%%%%%%%%%%%%%%%%%%%%%%%%%%%%
%%%%%%%%%%%%%%%%%%%%%%%%%%%%%%%%%%%%%%%%%%%%%%%%%%%%
%%%%%%%%%%%%%%%%%%%%%%%%%%%%%%%%%%%%%%%%%%%%%%%%%%%%
%%%%%%%%%%%%%%%%%%%%%%%%%%%%%%%%%%%%%%%%%%%%%%%%%%%%
\chapter{Instrumenting a simulation}\label{Instrumenting}

%==============================================================================%
%==============================================================================%
\section{Initializing and finalizing \Damaris{}}



This chapter will teach you how to instrument a simulation in order to use dedicated cores
with \Damaris{}.
\Damaris{} can be deployed in two different ways.
\begin{itemize}
	\item The dedicated cores and the simulation can be deployed separately (decoupled version). In
	this type of deployment the user has to be careful that exactly one \Damaris{} server is started on each node;
	\item \Damaris{} can be integrated in the MPI-based simulation and use part of its MPI processes, in which
	case the \Damaris{} library will take care of starting a server on each node and provide a communicator
	for client processes.
\end{itemize}
We warmly recommend the second solution. Even if it involves more modifications in the simulation's code,
it is the easiest way to deploy a \Damaris{}-instrumented simulation on most supercomputers.

\subsection{Minimal configuration file}\label{sec:MinimalConfigFile}

Before actually instrumenting your code, an XML configuration file has to be created.
This configuration file will contain some information about the simulation, 
the system, the data and the plugins that you want to use with \Damaris{}.
Listing~\ref{InitAndFinitXML} presents a minimal configuration file. This file will be necessary,
whatever deployment mode you choose (coupled or decoupled).

\noindent\begin{minipage}{\textwidth}
\vspace{0.5cm}
\lstset{language=XML,caption={\Damaris{} initial architecture description},label=InitAndFinitXML}
\lstinputlisting[language=XML]{listings/init.xml}
\end{minipage}

The \texttt{simulation}'s \texttt{name} is used in different backends, for example to name trace files or
to be added on figures. The language (\texttt{"c"}, \texttt{"cpp"}, or \texttt{"fortran"}) is optional (default is \texttt{"c"}),
it indicates in which language the simulation is written so fortran array layouts can be converted into
C layouts.
The \texttt{cores} tag provides the number of cores in a node (\texttt{cores count}) and the number of client
processes (\texttt{clients count}).

The \texttt{buffer} node provides the \texttt{name} and the \texttt{size} (in bytes) of the shared memory buffer that
will be used for the simulation processes to communicate with \Damaris{}.
The \texttt{queue} node provides the \texttt{name} and \texttt{size} 
(in number of messages) of the shared message queue.
Both buffer and queue can have an additional \texttt{type} attribute which can be either "posix" 
(by default) or "sysv".
POSIX shared memory will use the \function{shm\_open} function, while SYS-V shared memory will
use \function{shmget}. Many HPC operating systems don't have POSIX shared memory available.
Finally, an additional \texttt{blocks} attribute can be added to the buffer, taking an integer (e.g. \texttt{blocks="3"})
to indicate that several blocks of shared memory must be opened of the given \texttt{size}. This feature is
useful in order to open a shared memory buffer 
with a size bigger than the maximum allowed by the operating system (most operating systems limit this size
to 1 or 2~GB).
However, a variable can never have a size bigger than the blog size (since blocks may not be contiguous).
Finally the \texttt{data} and \texttt{actions} nodes will be described later.

\subsection{Decoupled version}

Now that a minimal configuration file has been provided, let's instrument our code.
This section will explain how to instrument the code for a decoupled deployment 
(\Damaris{} and the simulation are deployed as separate programs).

\subsubsection{Writing the code}

Listings~\ref{InitAndFinitC} and~\ref{InitAndFinitF90} present the minimal code instrumentation
for a C and a Fortran code respectively. All \Damaris{} function calls must be placed after a first call to the
\function{DC\_initialize} function (resp. \function{df\_initialize} in Fortran). This function
takes as parameter an XML configuration file 
and a MPI communicator.
The call to \function{DC\_initialize} should be placed after the MPI initialization, as it uses
MPI functions. \function{DC\_initialize} involves collective MPI communications: process 0 in
the provided communicator will load the XML file and broadcast its content to other processes
in the communicator, thus all the processes in the involved communicator should call this function.

Symmetrically a call
to \function{DC\_finalize} (resp. \function{df\_finalize}) ends all interactions with \Damaris{} before
the simulation finishes. Inside the simulation's main loop, a call to \function{DC\_end\_iteration}
(reps. \function{df\_end\_iteration}) will inform the dedicated cores that the current iteration has finished.
\Damaris{} keeps track of the iteration number internally: this number before the first call
to \function{DC\_end\_iteration} is 0.
A call to \function{DC\_end\_iteration} will update potential backends such as VisIt, 
and give more information to \Damaris{} about the average durations of an iteration.
As it involves collective communications in the dedicated cores, all the clients have to call
this function.
These three main functions are summarized in Table~\ref{tab:initFunctions}.

\begin{table}
\centering
\begin{tabular}{|l|}
	\hline
   \textbf{C functions} \\
   \hline
   \hline
   \function{int DC\_initialize(char* configfile, MPI\_Comm comm)}  \\
   \function{int DC\_end\_iteration()}  \\
   \function{int DC\_finalize()} \\
   \hline
   \hline
   \textbf{Fortran functions} \\
   \hline
   \hline
   \function{df\_initialize(character* configfile, MPI\_Fint comm, integer ierr)} \\
   \function{df\_end\_iteration(integer ierr)} \\
   \function{df\_finalize(integer ierr)} \\
   \hline
\end{tabular}\caption{Main \Damaris{} functions}\label{tab:initFunctions}
\end{table}

\noindent\begin{minipage}{\textwidth}
\lstset{language=C,caption={Initializing and finalizing \Damaris{}},label=InitAndFinitC}
\lstinputlisting[language=C]{listings/init.c}
\end{minipage}

\noindent\begin{minipage}{\textwidth}
\lstset{language=fortran,caption={Initializing and finalizing \Damaris{}},label=InitAndFinitF90}
\lstinputlisting[language=fortran]{listings/init.f90}
\end{minipage}

\textbf{Note:} \Damaris{} is not thread-safe yet. If your application uses an hybrid model
where different threads run on the same node, only one call to \function{DC\_initialize} should be
made and all the threads on a node should act as one single client (the XML configuration file
should also reflect this situation by setting the clients count accordingly).

\subsubsection{Compiling}

In order to compile your instrumented simulation, the \Damaris{} header files must be
provided, as well as the dependencies headers. The following options must be passed to the C compiler:

\begin{verbatim}
-I/path/to/damaris -I/path/to/dependencies/include
\end{verbatim}
Fortran program do not need any headers.
The linker will need the client library file \file{libdamaris.a} as well as other dependent libraries.
The following provides the necessary options for the linker:
\begin{verbatim}
-rdynamic
-L/path/to/damaris/client -ldamaris -L/path/to/dependencies/lib \
-Wl,--whole-archive,-ldamaris,--no-whole-archive \
-lxerces-c \
-lboost_program_options -lboost_filesystem -lboost_system -lboost_date_time \
-lboost_python -lpython2.6 \
-lsimV2 \
-lstdc++ -lrt -ldl
\end{verbatim}

The \texttt{-rdynamic} option instructs the linker to add all symbols, not only used ones, 
to the dynamic symbol table. 
This option is needed for some uses of dlopen in \Damaris{}. It is possible that the option
differs on some linkers.
Remove the Python-related dependencies if you didn't enabled the Python support.
Remove the simV2 line if you didn't enabled the VisIt support.

\subsubsection{Running}

Let's assume the program that we have compiled is called \file{sim}. As explained before,
the decoupled version of \Damaris{} requires two deployments.
First, go to the \file{server} subdirectory in the \Damaris{} source tree. If the compilation process
has finished correctly, a \file{server} program should be present. This is an MPI program that deploys
dedicated cores on every node. For now, we will use it on a single machine.
Call it using:
\begin{verbatim}
	./server --configuration=config.xml
\end{verbatim}
The server starts and waits for incoming data (this process is not a daemon). Additional options can be provided
to redirect standard outputs:
\begin{verbatim}
	./server --configuration=config.xml --stdout=out.txt --stderr=err.txt
\end{verbatim}

Now that our server is running, let's start the simulation using \texttt{./sim}. The simulation should correctly terminate. If the \textbf{INFO} flag was enabled when compiling \Damaris{}, the server will output a message
at the end of each iteration.

To shut down the server, just send a kill signal. The signal will be caught by the server, which will
properly clean its resources.

To run on multiple nodes, launch the server program with \texttt{mpirun}:
\begin{verbatim}
	mpirun -np X -machinefile FILE ./server --configuration=config.xml
\end{verbatim}
where X is the number of nodes and FILE contains the list of nodes. Be careful NOT to start several servers
on the same node! \Damaris{} is not yet able to handle multiple dedicated cores.

When killing the servers, MPI implementations usually send a kill -9 signal, which cannot be caught by
the servers. Thus, it may happen that shared memory objects remain opened and you have to delete them
manually (by deleting the corresponding files in /dev/shm for POSIX shared memory, or using ipcrm for
SYS-V shared memory). Many platforms automatically clean resources allocated by the user's program
such as shared memory objects.

\clearpage
\subsection{Integration in an MPI application}

\subsubsection{Writing the code}

An easier way of deploying an application using \Damaris{} consists in embedding the servers
directly within the simulation. Listings~\ref{InitMPIC} and \ref{InitMPIF90} show how to instrument
the code for this purpose.

The main difference with the decoupled version is the use of \function{DC\_mpi\_init\_and\_start} instead of
\function{DC\_initialize}. This function takes a communicator (typically MPI\_COMM\_WORLD)
and splits it such that the lowest rank on each node runs the server. Therefor, this
function does not returns for these ranks. It returns 1 for client processes, which enter the \emph{if}
statement. Clients then have to get back a valid communicator to communicate between each other.
This is possible thanks to the \function{DC\_mpi\_get\_client\_comm}.
Finally, the call to \function{DC\_kill\_server} at the end of the simulation will force the server to stop.
In server processes, \function{DC\_mpi\_init\_and\_start} will return 0 and thus, the \emph{if} block will not
be executed, leaving immediately the program.
\textbf{Note:} the servers will wait for \emph{all} the clients on a node to call \function{DC\_kill\_server} 
before stoping.

\noindent\begin{minipage}{\textwidth}
\vspace{0.5cm}
\lstset{language=C,caption={Integrating \Damaris{} in an MPI application},label=InitMPIC}
\lstinputlisting[language=C]{listings/mpiinit.c}
\end{minipage}

\noindent\begin{minipage}{\textwidth}
\lstset{language=fortran,caption={Integrating \Damaris{} in an MPI application},label=InitMPIF90}
\lstinputlisting[language=fortran]{listings/mpiinit.f90}
\end{minipage}

\subsubsection{Compiling}

Compiling a program in coupled deployment mode is pretty much the same than the decoupled version,
except that the \file{libdamaris-server.a} should be added when linking, using \texttt{-ldamaris-server}. 
This library must be added before all the other.

\subsubsection{Running}

This time the deployment is much simpler:

\begin{verbatim}
	mpirun -np X -machinefile FILE ./sim
\end{verbatim}

Where X is the number of processes (in total), and FILE lists the hostnames on which to start the program.
The \Damaris{} library will take care of starting one server per node. This time the machine file should
contain each hostname as many times as the number of cores in a node.

\textbf{Note:} \Damaris{} doesn't use any a priori knowledge about rank-to-cores binding and is able
to start one server per node even if process ranks are not consecutive in the node. It leverages the processor's
name to group processes, or platform specific information such as the host's personality on BG/P.

%==============================================================================%
%==============================================================================%
\section{Describing and Writing data}

Now that we have a working simulation instrumented with \Damaris{}, it is time to write data.
This section explains how to describe data in the XML file and how to send it to the dedicated cores.
All the data items must be described in the XML configuration within the \texttt{<data>} section.

\subsection{Parameters}

Parameters are simple values associated with a name and a type. It allows to
easily change some important value of the simulation without changing all the variables
and objects that depend on it. Parameters can also be used for the input of the simulation itself. 
Listing~\ref{parameterXML} shows how to define a parameter with a default value.
Parameters must be defined at the root of the \texttt{<data>}  section, not within nested groups
(later explained). We advice to describe the parameters at the beginning of the \texttt{<data>}  section,
even though other objects can be described that depend on a later defined parameter.

\noindent\begin{minipage}{\textwidth}
\vspace{0.5cm}
\lstset{language=XML,caption={Parameters descriptions},label=parameterXML}
\lstinputlisting[language=XML]{listings/parameter.xml}
\end{minipage}

Parameters must have a valid \texttt{name}, that is to say it should not contain any space or special symbols,
and should not start with a digit. The \texttt{type} attribute must be one of the types listed in 
Appendix~\ref{sec:types}. The provided \texttt{value} must be acceptable for the specified type. For instance
\texttt{value="xyz"} is not valid for an integer parameter.

Listing~\ref{parameterC} and~\ref{parameterF} show how to get and set parameters at run time
from the simulation. \textbf{Note:} when a process modifies a parameter, this modification is local and
not visible to other processes or to dedicated cores. The user is responsible for synchronizing all
processes when a parameter has to be globally changed.

\noindent\begin{minipage}{\textwidth}
\vspace{0.5cm}
\lstset{language=C,caption={Setting and getting parameters in C},label=parameterC}
\lstinputlisting[language=C]{listings/parameter.c}
\end{minipage}

\noindent\begin{minipage}{\textwidth}
\vspace{0.5cm}
\lstset{language=Fortran,caption={Setting and getting parameters in Fortran},label=parameterF}
\lstinputlisting[language=Fortran]{listings/parameter.f90}
\end{minipage}

\subsection{Layouts}

Layouts describe the shape of the data. \Damaris{} separates this description from the description
of the variables themselves since several variables can have the same layout.
A layout is characterized by a \texttt{name}, a basic \texttt{type} and a list of \texttt{dimensions}.
Two examples are given in Listing~\ref{layoutXML}. These descriptions
must be placed at the root of the \texttt{<data>} section, not in a nested group.

The \texttt{name} of a layout can be any string (may include special characters, white spaces, numbers, etc. even
though we advise to keep it simple and use classic C identifiers).
The \texttt{type} should be one of the basic types listed in Appendix~\ref{sec:types}.
Finally the list of dimensions is a comma-separated list of arithmetical expression featuring
+, -, *, /, \%, numbers and any defined parameters.
In the following examples, the first layout has three dimensions, the second one has two dimensions.

When a layout depends on parameters, any modification on the parameters will automatically modify
the layout. Like parameters, the layout is only locally affected and the modification is not propagated to
other processes. Making a layout depend on a parameter can be very useful for particle-based simulations,
where the number of particles is different at each process.

\noindent\begin{minipage}{\textwidth}
\vspace{0.5cm}
\lstset{language=XML,caption={Layouts descriptions},label=layoutXML}
\lstinputlisting[language=XML]{listings/layout.xml}
\end{minipage}

\textbf{Note:} \texttt{string} and \texttt{label} data types are variable-length types, thus
the dimension of the layout corresponds to the number of character that the variable can store.

\subsection{Variables and groups}

\subsubsection{XML description}

Actual data is described through the \texttt{<variable>} and \texttt{<group>} nodes, which
allow to build a hierarchy of variables. An example is given in Listing~\ref{varXML}.
Each variable must be given a \texttt{name}, and be associated to a defined \texttt{layout}.
These are the two mandatory attributes. The \texttt{time-varying} attributes indicates whether
a variable is expected to be written at every iteration, or just once at the beginning of the
simulation (i.e. before the first call to \function{DC\_end\_iteration}). 
The \texttt{visualizable} attribute indicates whether a variable is a primary object visualizable
by visualization backends (for instance, coordinate arrays are not visualizable). 
In the following example, we expect the \texttt{x\_coordinates} variable
to be the coordinate of points in a rectilinear grid. This data is not itself visualizable, however
the rectilinear grid (later described in Chapter~\ref{chap:VisIt}) will be a visualizable object.
Note that you can use a basic type instead of a layout, as for the \texttt{simple\_var} variable: 
basic types are already interpreted as layouts.

\noindent\begin{minipage}{\textwidth}
\vspace{0.5cm}
\lstset{language=XML,caption={Variables and groups descriptions},label=varXML}
\lstinputlisting[language=XML]{listings/var.xml}
\end{minipage}

\subsubsection{Writing full variables}

Now that the data is described, we can write it from our simulation. 
Listings~\ref{writingVarC} and~\ref{writingVarF90} present how to write a variable. 
The full name of the variable should be provided: this full name consists of a path within
the hierarchy of groups (similar to directories).

The returned value of this function (or the value contained in \texttt{ierr} in Fortran) is
0 in case of success, -1 if the variable does not exist, -2 if the library could not allocate
enough space in the shared memory segment (for instance if the shared memory is full),
and -3 indicates that the layout associated to the variable is unlimited, in which case \Damaris{}
cannot guess its size: use \function{DC\_chunk\_write} instead.

\noindent\begin{minipage}{\textwidth}
\vspace{0.5cm}
\lstset{language=C,caption={Writing a variable to \Damaris{} in C},label=writingVarC}
\lstinputlisting[language=C]{listings/write.c}
\end{minipage}

\noindent\begin{minipage}{\textwidth}
%\vspace{0.5cm}
\lstset{language=Fortran,caption={Writing a variable to \Damaris{} in Fortran},label=writingVarF90}
\lstinputlisting[language=Fortran]{listings/write.f90}
\end{minipage}

\subsubsection{Writing multiple domaines}

It may be necessary to write multiple domaines of variables instead of full variables. 
In order to do so, you have to specify the number of domains that a client is expected to
write, in the \texttt{clients} node of the XML file, as shown in Listing~\ref{writingBlocksXML}.

\noindent\begin{minipage}{\textwidth}
\vspace{0.5cm}
\lstset{language=XML,caption={Specifying multiple domains per client},label=writingBlocksXML}
\lstinputlisting[language=XML]{listings/blocks.xml}
\end{minipage}

This number is a maximum, a client may write less domains, but not more. Besides, if a client
writes $N$ domains, it must identify them $0$ to $N-1$.
Listings~\ref{writingBlocksC} and~\ref{writingBlocksF90} show how to write multiple blocks.
Each block is expected to have the size and shaped defined in the layout associated with the variable.

\noindent\begin{minipage}{\textwidth}
\vspace{0.5cm}
\lstset{language=C,caption={Writing a chunk of variable to \Damaris{} in C},label=writingBlocksC}
\lstinputlisting[language=C]{listings/blocks.c}
\end{minipage}

\noindent\begin{minipage}{\textwidth}
%\vspace{0.5cm}
\lstset{language=Fortran,caption={Writing a chunk of variable to \Damaris{} in Fortran},label=writingBlocksF90}
\lstinputlisting[language=Fortran]{listings/blocks.f90}
\end{minipage}

\subsubsection{Direct access to the shared memory}

The API presented above has the disadvantage of copying local data into the shared memory.
A more efficient way of proceeding consists in getting direct access to the shared memory.
Listings~\ref{allocateC} and~\ref{allocateF90} present this capability.

If \function{DC\_alloc} fails, it will return a NULL pointer. Otherwise, a valid pointer
to an allocated region of shared memory is returned. In Fortran, \function{df\_alloc} sets \texttt{ierr}
to 0 in case of failure, to non-zero otherwise.
After writing the data to the returned buffer, a call to \function{DC\_commit} will notify the server
of the presence of new data. After this call, the user is not expected to read or write the data anymore.
Equivalent functions (\function{DC\_alloc\_block} and \function{df\_alloc\_block}) exist to allocate blocks
when each client handles multiple domains.

\noindent\begin{minipage}{\textwidth}
\vspace{0.5cm}
\lstset{language=C,caption={Allocate a variable in shared memory from C},label=allocateC}
\lstinputlisting[language=C]{listings/alloc.c}
\end{minipage}

\noindent\begin{minipage}{\textwidth}
%\vspace{0.5cm}
\lstset{language=Fortran,caption={Allocate a variable in shared memory from Fortran},label=allocateF90}
\lstinputlisting[language=Fortran]{listings/alloc.f90}
\end{minipage}

Du to the C/Fortran interface, the use of \function{df\_alloc} is more complex in Fortran than in C.
The \Damaris{} module should be used (this module is located in the \file{client} directory of the \Damaris{}
source tree). An additional call to \function{c\_f\_pointer} is mandatory to convert the returned C pointer
to a valid Fortran array. This function takes a shape array to provide the extents along each dimensions.

Note that it can be desirable,
in order to implement efficient double-buffering, that a client waits some iterations before actually committing 
a variable. Different version of the function are available:
\begin{itemize}
\item \function{DC\_commit(const char* varname)} : \\ commits all the blocks of the current iteration;
\item \function{DC\_commit\_block(const char* varname, int32\_t block)} : \\ commit one specific block of the current iteration;
\item \function{DC\_commit\_iteration(const char* varname, int32\_t iteration)} : \\ commit all the block of a specific iteration;
\item \function{DC\_commit\_block\_iteration(const char* varname, int32\_t block, int32\_t iteration)} : \\ commit one specific block of a specific iteration.
\end{itemize}

Equivalent function are available in Fortran: \function{df\_commit},
\function{df\_commit\_block}, \function{df\_commit\_iteration} and \function{df\_commit\_block\_iteration}.
All take an \emph{ierr} integer in addition to the same parameters as the C functions.

%%%%%%%%%%%%%%%%%%%%%%%%%%%%%%%%%%%%%%%%%%%%%%%%%%%%
%%%%%%%%%%%%%%%%%%%%%%%%%%%%%%%%%%%%%%%%%%%%%%%%%%%%
%%%%%%%%%%%%%%%%%%%%%%%%%%%%%%%%%%%%%%%%%%%%%%%%%%%%
%%%%%%%%%%%%%%%%%%%%%%%%%%%%%%%%%%%%%%%%%%%%%%%%%%%%
%%%%%%%%%%%%%%%%%%%%%%%%%%%%%%%%%%%%%%%%%%%%%%%%%%%%
%%%%%%%%%%%%%%%%%%%%%%%%%%%%%%%%%%%%%%%%%%%%%%%%%%%%
\chapter{Writing and calling plugins}

Writing data to \Damaris{} is cool, but what should the server do with them? For now the shared memory
buffer will simply get quickly filled and future write operations will fail. The role of processing data
(and removing them from the shared memory) is delegated to plugins, which are triggered by events
sent by the simulation and performed asynchronously by the dedicated cores while the simulation keeps running.
This section explains how to program a plugin for \Damaris{} and to trigger it from the simulation.

%==============================================================================%
%==============================================================================%
\section{Describing actions}

The interactions between the simulation processes and the dedicated cores are based on events.
These events are sent by simulation processes to \Damaris{} using the shared message queue.
Like other \Damaris{} objects, these events have to be described in the XML configuration file.
These description should be placed in the \texttt{<actions>} section of the XML file.
Listing~\ref{actionXML} presents the description of such an \texttt{event}.
An \texttt{event} is described by a unique \texttt{name}, and an \texttt{action}, which corresponds to
the name of a C++ function. If this function is located in a shared \texttt{library} (.so file), the name
of the library has to be provided. \Damaris{} will look for shared libraries within the directories
specified by the LD\_LIBRARY\_PATH environment variable. If no library is specified, \Damaris{}
will look for the function within its own binary code. Be aware that, in decoupled deployment mode,
a dedicated core is not able to look for a function in the simulation's code.
Finally one important characteristic of events is the \texttt{scope}:
\begin{itemize}
	\item \textbf{"core"} indicates that every time a simulation process sends the event, the
	corresponding action is triggered. If the 16 processes of a 16-cores node 
	send the same event, the corresponding action is triggered 16 times.
	\item \textbf{"node"} indicates that \Damaris{} has to wait for all the cores of a node to send
	the same event (at the same iteration) before triggering the action. If the 16 processes
	send the event, the action is triggered only one. If one of the processes does not send the
	event, the corresponding action is never triggered.
	\item \textbf{"global"} (not yet implement) induces a particular behavior: each \Damaris{} 
	server will wait for all the clients of its node to send the event. It will then synchronize with
	all other dedicated cores to ensure that, eventually when all the clients of the simulation
	have sent the event, all the dedicated cores trigger the same action at the same time.
	Within a function called through a "global" event, MPI communications can safely be used.
\end{itemize}

\noindent\begin{minipage}{\textwidth}
\vspace{0.5cm}
\lstset{language=XML,caption={Actions description},label=actionXML}
\lstinputlisting[language=XML]{listings/action.xml}
\end{minipage}

The semantics of these scopes with respect to the order of triggered actions is the following:
\begin{itemize}
\item Two local ("core") events sent by the same process will be received in the same order by the dedicated core.
\item If an event (whatever the scope) is sent after writing or committing a variable, it is ensured that the
dedicated core has received the variable before triggering the event's action.
\item Two "node" events sent by all the processes of a node in the same order will trigger their actions in
the same order in the dedicated core. No particular order should be assumed if different processes send the
same event in a different order.
\item Two "global" events sent by all the processes of the simulation in the same order will trigger their actions
in the same order in all the dedicated cores. No particular order should be assumed if different processes send
the same event in a different order.
\item If a simulation process sends two events $e_1$ and $e_2$, if $e_1$ has a "core" scope 
then it is ensured that its corresponding action will be executed before the action of 
$e_2$. 
\item However, if $e_1$ has another scope than "core", no ordering can be assumed.
\item To force an ordered execution of $e_1$ and $e_2$ where the scope of $e_1$ is not "core",
a global synchronization barrier can be used in the simulation before sending $e_2$.
\end{itemize}

%==============================================================================%
%==============================================================================%
\section{\emph{Hello World!} plugin example}

Now let's take a look at a C++ function that can be called by \Damaris{}. An example of such a function
is provided in Listing~\ref{PluginCPP}. Note the use of \texttt{extern "C"} to prevent C++ name mangling.

\noindent\begin{minipage}{\textwidth}
\vspace{0.5cm}
\lstset{language=C++,caption={Example \Damaris{} plugin},label=PluginCPP}
\lstinputlisting[language=C++]{listings/plugin.cpp}
\end{minipage}

The first argument of such a function is a string representing the name of the event that triggered
the action (since several events can be connected to the same action). The second parameter
corresponds to the iteration at which the event has been sent. The third parameter is the rank
of the client that sent the event. For "node" and "global" actions, this source is set to $-1$.
Finally the last argument is only used by external plugins to send their own commands.

This first example only helps you start with plugins, more consideration on how to
access the data written by clients will be provided in Section~\ref{sec:internalAPI}.

%==============================================================================%
%==============================================================================%
\section{Compiling and linking C/C++ plugins}

To compile a plugin within a shared library, 
you need to create an object file from your source file with the -fPIC option,
then create a shared library:
\begin{verbatim}
g++ -fPIC -c something.cpp -I/path/to/damaris -I/path/to/boost/include
gcc -shared -Wl,-soname,libsomething.so -o libsomething.so something.o
\end{verbatim}

If you want to integrate your plugin in the application's code (for instance to avoid
loading the .so file from many processes at the same time, which could result in performance
degradation at large scale), simply compile the source of your plugin with your application. You must
dynamically link your simulation when compiling it.

%==============================================================================%
%==============================================================================%

\section{Sending events from the simulation}

To send events from the simulation, use the \function{DC\_signal} function (in C) or \function{df\_signal}
(in Fortran). These functions takes the name of an event as parameter; this name should correspond to
an event described in the configuration file.

\noindent\begin{minipage}{\textwidth}
\vspace{0.5cm}
\lstset{language=C,caption={Sending events from a C program},label=EventC}
\lstinputlisting[language=C]{listings/event.c}
\end{minipage}

\noindent\begin{minipage}{\textwidth}
%\vspace{0.5cm}
\lstset{language=fortran,caption={Sending events from a Fortran program},label=EventF90}
\lstinputlisting[language=fortran]{listings/event.f90}
\end{minipage}

The prototype of these two functions are summarized in Table~\ref{tab:signalFunctions}. The returned
value (or the ierr value in Fortran) is 0 in case of success, -1 in case of failure when sending the
event through the shared memory, and -2 if the name does not correspond to a defined event.

\begin{table}[h]
\centering
\begin{tabular}{|l|}
	\hline
   \textbf{C function} \\
   \hline
   \hline
   \function{int DC\_signal(char* configfile, int iteration)}  \\
   \hline
   \hline
   \textbf{Fortran function} \\
   \hline
   \hline
   \function{df\_signal(character* configfile, integer iteration, integer ierr)} \\
   \hline
\end{tabular}\caption{\Damaris{} signal functions}\label{tab:signalFunctions}
\end{table}

\section{Binding simulation's functions to events}

On some platforms, the \texttt{dlopen} function does not properly work, which causes \Damaris{} to not
be able to retrieve the plugin. In this case, another solution consists in providing the function's pointer
at run time (this works, of course, only in coupled mode).

To do so, \emph{all} the processes must call \texttt{DC\_bind\_function(``event name'',function\_ptr)} \emph{before}
calling \texttt{DC\_mpi\_init\_and\_start}, and in the \emph{same order} if multiple events are defined this way. 
The first argument to this function is the name of the event to define,
the second one is the pointer to a function that has the same signature than previously defined plugins.

This method is experimental, currently only available in C, and allows only a ``core'' scope. 
Besides, the event \emph{should not} be already defined in the configuration file.

%==============================================================================%
%==============================================================================%
\section{The internal \Damaris{} API}\label{sec:internalAPI}

In order to access the data from within a plugin, you will have to used \Damaris{}' data structures.
The main structure is the \texttt{VariableManager}, which can be used to access \texttt{Variable} instances
and, from there, \texttt{Chunk} instances (which correspond to our notion of ``domains''). 
We invite the reader to refer to the Doxygen documentation of
\Damaris{}, which we hope is complete enough to understand how to retrieve data,
and to post requests on the \Damaris{} mailing list if you need any help.


%==============================================================================%
%==============================================================================%
\chapter{Asynchronous vs Synchronous mode}

This is a small chapter to introduce the synchronous mode in \Damaris{}, a mode that was not initially
intended to be part of the software, but that we finally implemented in order to be able to compare
our design to time-partitioning approaches.

In synchronous mode, there is no dedicated core. Events are not anymore sent by simulation processes
through shared memory. Their action is instead triggered immediately and executed by the simulation
process itself.

\section{Enabling the synchronous mode}

The synchronous mode can be enabled simply by setting the number of clients to the number of cores
in the XML file, as exemplified in Listing~\ref{synchronousXML}.

\noindent\begin{minipage}{\textwidth}
\vspace{0.5cm}
\lstset{language=XML,caption={\Damaris{} XML file for synchronous mode},label=synchronousXML}
\lstinputlisting[language=XML]{listings/synch.xml}
\end{minipage}

Not that only simulations instrumented in coupled mode (i.e. with \function{DC\_mpi\_start}) 
can use this synchronous mode. This is because the server-side library must be available
in the simulation, as well as the functionalities it carries.

\section{Event scopes and semantics}

In synchronous mode, the simulation acts as if there were as many dedicated cores as clients.
In other words, each client has its own "dedicated core", and cannot access the data 
written by other processes.
Thus, "core", "node" and "global" scopes are not differentiated anymore.
Additionally, it is expected that if one process triggers a "global" action, all the process do,
since in a synchronous mode there is no way for a process to know whether other processes
will trigger the same action.

%%%%%%%%%%%%%%%%%%%%%%%%%%%%%%%%%%%%%%%%%%%%%%%%%%%%
%%%%%%%%%%%%%%%%%%%%%%%%%%%%%%%%%%%%%%%%%%%%%%%%%%%%
%%%%%%%%%%%%%%%%%%%%%%%%%%%%%%%%%%%%%%%%%%%%%%%%%%%%
%%%%%%%%%%%%%%%%%%%%%%%%%%%%%%%%%%%%%%%%%%%%%%%%%%%%
%%%%%%%%%%%%%%%%%%%%%%%%%%%%%%%%%%%%%%%%%%%%%%%%%%%%
%%%%%%%%%%%%%%%%%%%%%%%%%%%%%%%%%%%%%%%%%%%%%%%%%%%%
\chapter{Python interface}\label{Python}

\Damaris{} allows plugins written in Python, as soon as the Python system has been
enabled when compiling \Damaris{}. This section explains how to install and use the Python
plugins system.

%==============================================================================%
%==============================================================================%
\section{Installation requirements}

Download Python (\Damaris{} has been tested with version 2.6.8 so far, even though any 2.x version
should work. Because of compatibility issues with Boost, it is not ensured that version 3.x will work) at 
\url{http://www.python.org/download/releases/}.
Untar and install using
\begin{verbatim}
./configure --prefix=$HOME/local --disable-shared
make
make install
\end{verbatim}

This will install Python in your \installdir directory. \textbf{Important note:} if you have installed VisIt,
VisIt provides a local python installation. In order to avoid conflicts when using both Python and VisIt,
use the one provided by VisIt (located in \texttt{/path/to/visit/python}).

\Damaris{} also needs NumPy, from \url{http://www.scipy.org/Download/}. Download the source archive,
then untar it install it:

\begin{verbatim}
python setup.py install
\end{verbatim}
\textbf{Note:} make sure the \textbf{python} command you runs calls the Python interpreter you just installed
or the one from VisIt, and NOT the default interpreter of your platform.

Now that Python and NumPy are both installed, recompile \Damaris{} (make sure to change
the lines in the CMakeLists.txt that include PYTHON\_ROOT and NUMPY\_INCLUDE\_DIR).

\section{XML configuration}

In the XML configuration of your simulation, you can provide some information related to
Python, by writing the lines in Listing~\ref{PythonXML} after the \texttt{</actions>} tag.

\noindent\begin{minipage}{\textwidth}
\vspace{0.5cm}
\lstset{language=XML,caption={Providing Python options},label=PythonXML}
\lstinputlisting[language=XML]{listings/pythonpath.xml}
\end{minipage}

The \texttt{<path>} section corresponds to the PYTHON\_PATH environment variable,
while the \texttt{<home>} section corresponds to the PYTHON\_HOME environment variable.
These options are useful as it may be difficult on some machines to properly forward environment
variables to compute nodes.

%==============================================================================%
%==============================================================================%
\section{Python scripts as plugins}

Python plugins in \Damaris{} are written in Python scripts. To connect an event to
a script, follow the example in Listing~\ref{PluginPythonXML}. By calling \function{DC\_signal}
with the name provided in the configuration, the corresponding script will be executed.
\Damaris{} currently support only Python scripts, thus the \texttt{language} tag will always contain
"python". The \texttt{scope} tat has the same semantic than for events.

\noindent\begin{minipage}{\textwidth}
\vspace{0.5cm}
\lstset{language=XML,caption={Describing a Python script},label=PluginPythonXML}
\lstinputlisting[language=XML]{listings/python.xml}
\end{minipage}

%==============================================================================%
%==============================================================================%
\section{Accessing \Damaris{} data from Python}

In the Python scripts, everything related to \Damaris{} is located in the \texttt{damaris} module. 
Use \texttt{import damaris} and \texttt{import numpy} to access these functionalities.
Table~\ref{tab:pythonAPI} presents how to access chunks of data from Python.

\begin{table}[h]
\centering{}
   \begin{tabular}{|l|l|l|}
       \hline
       Statement & Description \\
       \hline
       \hline
       damaris.iteration & Iteration at which the event has been sent. \\
       damaris.source & Rank of the process that sent the event.\\
       damaris.clear & Remove all the chunks from all the variables. \\
       var = damaris.open("group/varname") & Opens a variable.\\
       var.name & Name of the variable.\\
       var.fullname & Full name (including groups) of the variable.\\
       var.unit & Unit of the variable.\\
       var.description & Description from the XML file. \\
       layout = var.layout & Layout of the variable.\\
       list = var.select(\{"iteration": x, "source": y\}) & Select a list of chunks by source and/or iteration.\\
       var.remove(c) & Removes a Chunk (free the memory). \\
       var.clear() & Removes all the chunks attached to the variable. \\
       layout.name & Name of the layout. \\
       layout.type & Type of the layout (string). \\
       layout.extents & Array of dimensions. \\
       chunk.source & Source of the Chunk (process rank). \\
       chunk.iteration & Iteration of the Chunk. \\
       chunk.type & Type of the chunk (string). \\
       chunk.lower\_bounds & List of lower bounds along each dimension. \\
       chunk.upper\_bounds & List of upper bounds along each dimension. \\
       chunk.data & NumPy array of the chunk's data. \\
       \hline
   \end{tabular}\caption{\Damaris{} Python API}\label{tab:pythonAPI}
\end{table}


%%%%%%%%%%%%%%%%%%%%%%%%%%%%%%%%%%%%%%%%%%%%%%%%%%%%
%%%%%%%%%%%%%%%%%%%%%%%%%%%%%%%%%%%%%%%%%%%%%%%%%%%%
%%%%%%%%%%%%%%%%%%%%%%%%%%%%%%%%%%%%%%%%%%%%%%%%%%%%
%%%%%%%%%%%%%%%%%%%%%%%%%%%%%%%%%%%%%%%%%%%%%%%%%%%%
%%%%%%%%%%%%%%%%%%%%%%%%%%%%%%%%%%%%%%%%%%%%%%%%%%%%
%%%%%%%%%%%%%%%%%%%%%%%%%%%%%%%%%%%%%%%%%%%%%%%%%%%%
\chapter{VisIt interface}\label{chap:VisIt}

\Damaris{} can be used to perform in-situ visualization with VisIt through its \emph{libsimV2} interface, 
both in synchronous and asynchronous modes. \textbf{Warning:} because of a bug in VisIt (as of version 2.5.1),
The VisIt backend only works in decoupled mode, or in synchronous mode (not in asynchronous coupled).
A patch has been sent to the VisIt developers and future versions of VisIt should work with the
synchronous coupled mode of \Damaris{}.
VisIt must be installed on your platform (see the VisIt website at \url{https://wci.llnl.gov/codes/visit/}).
Follow the installation process of \Damaris{} presented in Chapter~\ref{chap:downloadingAndInstalling}
and enable VisIt by providing the VISIT\_ROOT 
VisIt install directory in the CMakeLists.txt (see Table~\ref{tab:cmake}).
\Damaris{} requires VisIt 2.5.0 or greater.

%==============================================================================%
%==============================================================================%
\section{VisIt options}

In a way similar to Python, some options can be provided to VisIt through the XML file.
Listing~\ref{VisItXML} presents these options. The \texttt{<path>} tag locates the installation
directory of VisIt, it is used by the simulation to load VisIt plugins.
The \texttt{<options>} tag provides an equivalent to VisIt's command line options (see
the VisIt manual). Most of the time, the \texttt{<options>} section is not required.

\noindent\begin{minipage}{\textwidth}
\vspace{0.5cm}
\lstset{language=XML,caption={Providing VisIt options},label=VisItXML}
\lstinputlisting[language=XML]{listings/visit.xml}
\end{minipage}

%==============================================================================%
%==============================================================================%
\section{Meshes and fields description}

\subsection{Meshes}

The current version of \Damaris{} only supports three types of objects exposed to VisIt:
rectilinear meshes, curvilinear meshes and commands. This section shows how to expose
a mesh and how to connect a variable to a mesh.
Just like other \Damaris{} objects, meshes have to be described in the XML file, as shown
in Listing~\ref{MeshVisItXML}. This kind of description should be placed in the \texttt{<data>} section
of the configuration, at the root or within any subgroup.

\noindent\begin{minipage}{\textwidth}
\vspace{0.5cm}
\lstset{language=XML,caption={Describing a Mesh in \Damaris{}},label=MeshVisItXML}
\lstinputlisting[language=XML]{listings/mesh.xml}
\end{minipage}

A mesh is characterized by a \texttt{name}, a \texttt{type} (currently only "rectilinear" or "curvilinear" are
allowed), and a topological dimension (for instance a surface in a 3D space is characterized by three
coordinates for each of its points, but has a 2D topology).

Finally two to three coordinate have to be provided through the \texttt{<coord>} tag. 
These coordinates refer to existing variables from the configuration. In the current version of \Damaris{},
the name of these variables should be absolute (a mesh is not capable to look for a variable only in
its group). The \texttt{unit} of a coordinate, when provided, hides the \texttt{unit} of the corresponding variable.
The \texttt{label} is used when drawing figures.

\subsection{Fields}

A field is a value that can be placed on each node or each zone of a mesh.
To allow a described variable to be a field, some new attributes have to be added in its description.
Listing~\ref{xml:field} shows the description of a variable with attributes related to visualization.

\noindent\begin{minipage}{\textwidth}
\vspace{0.5cm}
\lstset{language=XML,caption={Extended variable description to expose it as a field},label=xml:field}
\lstinputlisting[language=XML]{listings/field.xml}
\end{minipage}

The most important attribute is \texttt{mesh}, which is a reference to an existing mesh on which
to draw the field. The \texttt{centering} attribute can be either "nodal" (values correspond to nodes 
of the mesh) or "zonal" (values correspond to zones in the mesh). The \texttt{type} should be
"scalar" (other types are not yet supported by \Damaris{}).

By default, \texttt{visualizable} and \texttt{time-varying} are already set to "true", \texttt{unit} can be omitted.
The other attributes (\texttt{type} and \texttt{centering}) can be omitted if the variable is not concerned by
visualization.
The ``rule'' is that variables that serve as coordinates in meshes are not visualizable, while
other variable are considered as fields and are visualizable, thus must have an attached mesh.
%==============================================================================%
%==============================================================================%
\section{External actions and interactivity}

Events and scripts can be set as ``external'' by adding the \texttt{external="true"} attribute to their
description. An external event can be fired from the VisIt client itself. These events are exposed as commands
and are accessible through the VisIt viewer's commands interface.
Clicking on a command will trigger the corresponding event in all dedicated processes (or all processes
in synchronous mode). It is ensured that all the actions will be triggered at the same time, thus
the actions can safely use global communications using MPI.

%%%%%%%%%%%%%%%%%%%%%%%%%%%%%%%%%%%%%%%%%%%%%%%%%%%%
%%%%%%%%%%%%%%%%%%%%%%%%%%%%%%%%%%%%%%%%%%%%%%%%%%%%
%%%%%%%%%%%%%%%%%%%%%%%%%%%%%%%%%%%%%%%%%%%%%%%%%%%%
%%%%%%%%%%%%%%%%%%%%%%%%%%%%%%%%%%%%%%%%%%%%%%%%%%%%
%%%%%%%%%%%%%%%%%%%%%%%%%%%%%%%%%%%%%%%%%%%%%%%%%%%%
%%%%%%%%%%%%%%%%%%%%%%%%%%%%%%%%%%%%%%%%%%%%%%%%%%%%

\appendix
\chapter{\Damaris{} basic types}\label{sec:types}

\begin{table}[h]
\centering
   \begin{tabular}{|c|c|c|l|}
      \hline
      \Damaris{} type & C type & Fortran type & Description \\
      \hline
short & short int & integer*2 &
                    2 bytes integer value.\\
int & int & integer &
                    4 bytes integer value.\\
integer & int & integer &
                    4 bytes integer value.\\
long & long int & integer*8 &
                    8 bytes integer value.\\
float & float & real*4 &
                    4 bytes float value.\\
real & float & real*4 &
                    4 bytes float value.\\
double & double & real*8 &
                    8 bytes float value.\\
char & char & character &
                    1 byte character value.\\
character & char & character &
                    1 byte character value.\\
string & char[] & character(:) &
                    Variable length null-terminated string.\\
label & ? & ? &
                    Not implemented yet.\\
                    \hline
   \end{tabular}
   \caption{Correspondence between \Damaris{} types, C and Fortran types.}
\end{table}

\printindex

\end{document}
