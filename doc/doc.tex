\documentclass[11pt]{report} 
\usepackage{url}
\usepackage{anysize}
\usepackage{makeidx}
\usepackage{hyperref}
\usepackage{listings}
\usepackage{color}

\marginsize{2cm}{2cm}{2cm}{2cm}
 
\definecolor{dkgreen}{rgb}{0,0.6,0}
\definecolor{mauve}{rgb}{0.58,0,0.82}
\definecolor{gray}{rgb}{0.4,0.4,0.4}
\definecolor{darkblue}{rgb}{0.0,0.0,0.6}
\definecolor{cyan}{rgb}{0.0,0.0,1}
 
\lstset{ %
  language=C,                % the language of the code
  basicstyle=\footnotesize,           % the size of the fonts that are used for the code
  %numbers=left,                   % where to put the line-numbers
  %numberstyle=\tiny\color{black},  % the style that is used for the line-numbers
  %stepnumber=1,                   % the step between two line-numbers. If it's 1, each line 
                                  % will be numbered
  numbersep=5pt,                  % how far the line-numbers are from the code
  backgroundcolor=\color{white},      % choose the background color. You must add \usepackage{color}
  showspaces=false,               % show spaces adding particular underscores
  showstringspaces=false,         % underline spaces within strings
  showtabs=false,                 % show tabs within strings adding particular underscores
  frame=single,                   % adds a frame around the code
  frameround=ftft,
  float=th,
  rulecolor=\color{black},        % if not set, the frame-color may be changed on line-breaks within not-black text (e.g. commens (green here))
  tabsize=2,                      % sets default tabsize to 2 spaces
  captionpos=b,                   % sets the caption-position to bottom
  breaklines=true,                % sets automatic line breaking
  breakatwhitespace=false,        % sets if automatic breaks should only happen at whitespace
  title=\lstname,                   % show the filename of files included with \lstinputlisting;
                                  % also try caption instead of title
  keywordstyle=\color{blue},          % keyword style
  commentstyle=\color{dkgreen},       % comment style
  stringstyle=\color{mauve},         % string literal style
  escapeinside={\%*}{*)},            % if you want to add a comment within your code
  morekeywords={*,...}               % if you want to add more keywords to the set
}

\lstdefinelanguage{XML}
{
  morestring=[b]",
  %morestring=[s]{>}{<},
  morecomment=[s]{<?}{?>},
  morecomment=[s]{<!--}{-->},
  stringstyle=\color{mauve},
  identifierstyle=\color{blue},
  keywordstyle=\color{blue},
  morekeywords={xmlns,version,type,simulation,architecture,
  cores,clients,buffer,queue,data,actions,parameter,layout,variable,
  group}% list your attributes here
}

\newcommand{\Damaris}{\emph{\textbf{Damaris}}}
\newcommand{\installdir}[1]{\texttt{\$HOME/local#1}}
\newcommand{\file}[1]{\emph{#1}}
\newcommand{\function}[1]{\texttt{#1}}

\makeindex

%%%%%%%%%%%%%%%%%%%%%%%%%%%%%%%%%%%%%%%%%%%%%%%%%%%%
%%%%%%%%%%%%%%%%%%%%%%%%%%%%%%%%%%%%%%%%%%%%%%%%%%%%
%%%%%%%%%%%%%%%%%%%%%%%%%%%%%%%%%%%%%%%%%%%%%%%%%%%%
\begin{document}

\title{\Huge{ 
	 Getting Started with \Damaris{}} \\
	\normalsize{} version 1.0 \emph{``thesis release''}}
\author{\textbf{Matthieu Dorier}}
\date{\today}
\maketitle

\setcounter{tocdepth}{1}
\tableofcontents

%%%%%%%%%%%%%%%%%%%%%%%%%%%%%%%%%%%%%%%%%%%%%%%%%%%%
%%%%%%%%%%%%%%%%%%%%%%%%%%%%%%%%%%%%%%%%%%%%%%%%%%%%
%%%%%%%%%%%%%%%%%%%%%%%%%%%%%%%%%%%%%%%%%%%%%%%%%%%%
%%%%%%%%%%%%%%%%%%%%%%%%%%%%%%%%%%%%%%%%%%%%%%%%%%%%
%%%%%%%%%%%%%%%%%%%%%%%%%%%%%%%%%%%%%%%%%%%%%%%%%%%%
%%%%%%%%%%%%%%%%%%%%%%%%%%%%%%%%%%%%%%%%%%%%%%%%%%%%
\chapter*{The \Damaris{} approach}

Large-scale simulations usually perform I/O in a time-partitioning manner.
They periodically stop and all processes perform a write (for checkpointing
or future analysis purpose) concurrently to a parallel file system. 
This approach has also been used to perform in situ visualization, 
where the simulation periodically stops to compute images.

On recent infrastructures, the large number of cores available in each SMP node makes
it possible to offload these tasks in order to gain efficiency by working 
in a space-partitioning manner.
\Damaris{} proposes to dedicate a subset of cores on each SMP node to perform these tasks,
or even to dedicate entire nodes.
In dedicated-core mode, it uses shared memory to communicate data from the cores 
running the simulation's code and the cores dedicated to these set-aside,
data intensive tasks.

\Damaris{} can be enriched with new functionalities through a plugin system. 
Plugins can be written in C++ or Python, making asynchronous, 
in-situ data management straightforward for end-users.
Besides, \Damaris{} can be connected to the VisIt visualization software to provide
non-impacting, in situ, interactive visualization to existing simulations.

Version 1.0 of \Damaris{} corresponds to the version released by its author,
Matthieu Dorier, at the end of his PhD thesis.

%%%%%%%%%%%%%%%%%%%%%%%%%%%%%%%%%%%%%%%%%%%%%%%%%%%%
%%%%%%%%%%%%%%%%%%%%%%%%%%%%%%%%%%%%%%%%%%%%%%%%%%%%
%%%%%%%%%%%%%%%%%%%%%%%%%%%%%%%%%%%%%%%%%%%%%%%%%%%%
%%%%%%%%%%%%%%%%%%%%%%%%%%%%%%%%%%%%%%%%%%%%%%%%%%%%
%%%%%%%%%%%%%%%%%%%%%%%%%%%%%%%%%%%%%%%%%%%%%%%%%%%%
%%%%%%%%%%%%%%%%%%%%%%%%%%%%%%%%%%%%%%%%%%%%%%%%%%%%
\chapter{Downloading and Installing}\label{chap:downloadingAndInstalling}

%%%%%%%%%%%%%%%%%%%%%%%%%%%%%%%%%%%%%%%%%%%%%%%%%%%%
%%%%%%%%%%%%%%%%%%%%%%%%%%%%%%%%%%%%%%%%%%%%%%%%%%%%
%%%%%%%%%%%%%%%%%%%%%%%%%%%%%%%%%%%%%%%%%%%%%%%%%%%%
%%%%%%%%%%%%%%%%%%%%%%%%%%%%%%%%%%%%%%%%%%%%%%%%%%%%
%%%%%%%%%%%%%%%%%%%%%%%%%%%%%%%%%%%%%%%%%%%%%%%%%%%%
%%%%%%%%%%%%%%%%%%%%%%%%%%%%%%%%%%%%%%%%%%%%%%%%%%%%


%==============================================================================%
%==============================================================================%
\section{Dependencies}\label{sec:dependencies}

In the following, we will assume that you install all dependencies locally in \installdir.
\Damaris{} requires the following dependencies.
\begin{itemize}
	\item CMake (2.8 or newer), \\ available at \url{http://www.cmake.org/}
	\item The XSD code synthesis library, \\ available at \url{http://www.codesynthesis.com/products/xsd/}
	\item The Xerces-C library (version 3.1.1 or newer), \\
	available at \url{http://xerces.apache.org/xerces-c/download.cgi}
	\item The following parts of the Boost C++ library (version 1.51 or newer): 
	date\_time, filesystem and system, available at \url{http://www.boost.org/}
	\item An MPI C++ compiler.
\end{itemize}

The support for VisIt can be enabled by installing the corresponding
libraries (see Chapters~\ref{chap:VisIt}).

\vspace{0.5cm}

\noindent\textbf{Note:} The Python plugin system, enabled in earlier versions of \Damaris{}
and presented in some of the papers related to \Damaris,
has been disabled starting from version 1.0, the reason being the lack of engineering 
support in our team and the lack of users effectively leveraging Python. If you are
interested in using Python plugins with \Damaris, we invite you to contact the developers
and the necessary work will be carried out for the next version.

\subsection{Xerces-C}

Download Xerces-C (version 3.1.1 or greater). Take the source for your platform.
Configure and compile it:

\begin{verbatim}
./configure --prefix=$HOME/local --disable-threads --disable-network
make
make install
\end{verbatim}

\subsection{Boost}

Download Boost (version 1.51 or higher). Configure and compile it:

\begin{verbatim}
./bootstrap.sh --prefix=$HOME/local --with-libraries=date_time,system,filesystem 
./b2
./b2 install
\end{verbatim}

\subsection{XSD codesynthesis}

Download the XSD sources for your platform.
Untar/unzip it and install it by copying it into the \installdir directory.
Make sure that the \texttt{xsd} executable is properly located in the \installdir/bin
directory and that this directory is in the PATH environment variable (this is required
when compiling \Damaris{} to generate the XML parser). Make sure that the \texttt{xsd} directory
in the XSD distribution is copied into the \installdir/include directory.

%==============================================================================%
%==============================================================================%

\section{Compiling \Damaris{}}

Download and untar the \Damaris{} archive. \Damaris{} uses CMake to be compiled.
Make sure the \texttt{cmake} command is in your PATH environment variable.

In the \Damaris{} root directory, the \file{CMakeLists.txt} file contains the 
compilation directives and various options. 
If needed, you can open this file and modify the lines listed in 
Table~\ref{tab:cmake} to adapt the location of dependencies.
When these modifications are done, make sure your working directory is \Damaris' root 
directory and type
\begin{verbatim}
cmake -G "Unix Makefiles" -DCMAKE_INSTALL_PREFIX:PATH=$HOME/local
\end{verbatim}
This will generate the \file{Makefile} to compile \Damaris. Go on with 
\begin{verbatim}
make
make install
\end{verbatim}
You're done! Check your \installdir, you should see \texttt{Damaris.h} in the 
include directory, as well as a \texttt{damaris} directory,
and \texttt{libdamaris.a} in the lib directory. If \Damaris{} has been compiled 
with Fortran support, a \texttt{damaris.mod} file is also present in 
the target install directory.

\begin{table}
\centering
\begin{tabular}{|l|l|l|}
	\hline
   Line & Variable & Content \\
   \hline
   \hline
   13 &  & Name of the C++ MPI compiler. \\
   14 &  & Name of the C MPI compiler. \\
   15 &  & Name of the Fortran MPI compiler (uncomment if needed). \\
   27 & EXTERNAL\_ROOT & Prefix path where dependencies have been installed. \\
   28 & XERCESC\_ROOT & Xerces-C library location. \\
   29 & XSD\_ROOT & XSD library location. \\
   30 & BOOST\_ROOT & Boost library location. \\
   31 & VISIT\_ROOT & Location of VisIt. Comment if you don't need VisIt support. \\
   61 & BGP & Uncomment if you run on BlueGene/P. \\
   \hline
\end{tabular}\caption{CMakeLists.txt modifications before compiling \Damaris{}}\label{tab:cmake}
\end{table}

\vspace{0.5cm}

\noindent\textbf{Note:} \Damaris{} uses MPI\_Get\_processor\_name and expect it to return the same
name on all the cores of the same SMP node. It may not be the case on some platforms. 
On BlueGene/P, don't forget to use the BGP flag to use BGP's personality functions instead.



%%%%%%%%%%%%%%%%%%%%%%%%%%%%%%%%%%%%%%%%%%%%%%%%%%%%
%%%%%%%%%%%%%%%%%%%%%%%%%%%%%%%%%%%%%%%%%%%%%%%%%%%%
%%%%%%%%%%%%%%%%%%%%%%%%%%%%%%%%%%%%%%%%%%%%%%%%%%%%
%%%%%%%%%%%%%%%%%%%%%%%%%%%%%%%%%%%%%%%%%%%%%%%%%%%%
%%%%%%%%%%%%%%%%%%%%%%%%%%%%%%%%%%%%%%%%%%%%%%%%%%%%
%%%%%%%%%%%%%%%%%%%%%%%%%%%%%%%%%%%%%%%%%%%%%%%%%%%%
\chapter{Instrumenting a Simulation}\label{Instrumenting}


%==============================================================================%
%==============================================================================%
\section{Initializing and finalizing \Damaris{}}

This chapter will teach you how to instrument a simulation in order to 
use \Damaris. \Damaris{} requires the simulation to be based on MPI.

\subsection{Minimal configuration file}\label{sec:MinimalConfigFile}

Before actually instrumenting your code, an XML configuration file has 
to be created. This configuration file will contain some information about 
the simulation, the system, the data and the plugins that you want to use 
with \Damaris. Listing~\ref{InitAndFinitXML} presents a minimal configuration file.

\noindent\begin{minipage}{\textwidth}
\vspace{0.5cm}
\lstset{language=XML,caption={\Damaris{} initial architecture description},label=InitAndFinitXML}
\lstinputlisting[language=XML]{listings/init.xml}
\end{minipage}

The \texttt{simulation}'s \texttt{name} is used in different backends, for example to name trace files or
to be added on figures when using in situ visualization. The language (\texttt{"c"}, \texttt{"cpp"}, 
or \texttt{"fortran"}) is optional (default is \texttt{"c"}), it indicates in which language the simulation 
is written, so that fortran array layouts can be converted into C layouts.
The \texttt{domain} section provides the number of domains that a single process will handle. Some simulations
indeed split their variables into blocks (or domains) and distribute several blocks per process. If the number of
blocks is not the same on every process, it should correspond to the maximum number of blocks that a single process
may have to handle.
The \texttt{dedicated} section provides the number of dedicated cores in a node, and the number of dedicated nodes.
These different configurations are explained in more details in Chapter~\ref{modes}.

The \texttt{buffer} section provides the \texttt{name} and the \texttt{size} (in bytes) of 
the buffer that will be used for the simulation processes to communicate with the dedicated cores
or dedicated nodes. It can take an additional \texttt{type} attribute which can be either ``posix'' 
(by default) or ``sysv'', representing the type of share memory to be used.
POSIX shared memory will use the \function{shm\_open} function, while SYS-V shared memory will
use \function{shmget}. Some HPC operating systems indeed don't have one or the other type available.
Finally, an additional \texttt{blocks} attribute can be added to the buffer, 
taking an integer (e.g. \texttt{blocks="3"}) to indicate that several blocks of shared memory 
must be opened of the given \texttt{size}. This feature is useful in order to open a shared memory buffer 
with a size bigger than the maximum allowed by the operating system 
(most operating systems limit this size to 1 or 2~GB).
However, a variable can never have a size bigger than the block size 
(since blocks may not be contiguous).

The \texttt{queue} section provides the \texttt{name} and \texttt{size} 
(in number of messages) of the message queues
used to communicate between clients and dedicated resources. This is the remnant of
older versions of \Damaris{} in which the queue was handled in shared memory as well.
Simply keeping the default name and size works just fine.

Finally the \texttt{data} and \texttt{actions} sections will be described later.

\subsection{Instrumenting a simulation}

Now that a minimal configuration file has been provided, let's instrument our code.

\subsubsection{Writing the code}

Listings~\ref{InitAndFinitC} and~\ref{InitAndFinitF90} present the minimal code instrumentation
for a C and a Fortran code respectively. All \Damaris{} function calls must be placed after a first call to the
\function{damaris\_initialize} function (resp. \function{damaris\_initialize\_f} in Fortran). This function
takes as parameter an XML configuration file 
and a MPI communicator (generally MPI\_COMM\_WORLD).
The call to \function{damaris\_initialize} should be placed after the MPI initialization, as it uses
MPI functions. \function{damaris\_initialize} involves collective MPI communications: process 0 in
the provided communicator will load the XML file and broadcast its content to other processes
in the communicator, thus all the processes in the involved communicator should call this function.

Symmetrically a call
to \function{damaris\_finalize} (resp. \function{damaris\_finalize\_f}) frees the resources used by
\Damaris{} before the simulation finishes. It should be called before finalizing MPI.

These functions only prepare \Damaris{} but do not start the dedicated cores or nodes.
To start the dedicated resources, a call to \function{damaris\_start} (resp. \function{damaris\_start\_f})
is necessary. This function takes a pointer to an integer as parameter. \Damaris{} will automatically
analyze the platform on which the simulation runs and select cores to be either clients or servers.
On client processes, this function returns and the integer is set to a positive value. On server processes,
this function blocks and runs the server loop. It will only return when the clients have all called 
\function{damaris\_stop} (resp. \function{damaris\_stop\_f}), and the integer will be set to 0.
This integer can thus be used to select whether or not to run the simulation loop on a particular process.

Inside the simulation's main loop, a call to 
\function{damaris\_client\_comm\_get} gives a
communicator that the clients can use to communicate with one another (i.e., a replacement for
the MPI\_COMM\_WORLD that now includes dedicated resources). \textbf{It is crucial that the
simulation only uses this communicator and not MPI\_COMM\_WORLD, as global communicator}. 

\function{damaris\_end\_iteration}
(resp. \function{damaris\_end\_iteration\_f}) informs the dedicated cores that the current iteration 
has finished. \Damaris{} keeps track of the iteration number internally: 
this number is equal to 0 before the first call to \function{damaris\_end\_iteration}.
A call to \function{damaris\_end\_iteration} will update potentially connected backends such as VisIt.
As it involves collective communications, all the clients have to call this function.
These main functions are summarized in Table~\ref{tab:initFunctions}.

\begin{table}
\centering
\begin{tabular}{|l|}
	\hline
   \textbf{C functions} \\
   \hline
   \hline
   \function{int damaris\_initialize(const char* configfile, MPI\_Comm comm)}  \\
   \function{int damaris\_start(int* is\_client)} \\
   \function{int damaris\_client\_comm\_get(MPI\_Comm* comm)} \\
   \function{int damaris\_end\_iteration()}  \\
   \function{int damaris\_stop()} \\
   \function{int damaris\_finalize()} \\
   \hline
   \hline
   \textbf{Fortran functions} \\
   \hline
   \hline
   \function{damaris\_initialize\_f(character* configfile, MPI\_Fint comm, integer ierr)} \\
   \function{damaris\_start\_f(integer is\_client, integer ierr)} \\
   \function{damaris\_client\_comm\_get\_f(MPI\_Fint comm, integer ierr)} \\
   \function{damaris\_end\_iteration\_f(integer ierr)} \\
   \function{damaris\_stop\_f(integer ierr)} \\
   \function{damaris\_finalize\_f(integer ierr)} \\
   \hline
\end{tabular}\caption{Main \Damaris{} functions}\label{tab:initFunctions}
\end{table}

\noindent\begin{minipage}{\textwidth}
\lstset{language=C,caption={Initializing and finalizing \Damaris{}},label=InitAndFinitC}
\lstinputlisting[language=C]{listings/init.c}
\end{minipage}

\noindent\begin{minipage}{\textwidth}
\lstset{language=fortran,caption={Initializing and finalizing \Damaris{}},label=InitAndFinitF90}
\lstinputlisting[language=fortran]{listings/init.f90}
\end{minipage}

\textbf{Note:} \Damaris{} is not thread-safe yet. If your application uses an hybrid model
where different threads run on the same node, only one thread should call \function{damaris\_initialize} 
(and other \Damaris{} functions), and all the threads on a node should act as one single client.

\subsubsection{Compiling}

In order to compile your instrumented simulation, the \Damaris{} header files must be
provided, as well as the dependencies' headers. The following options must be passed to the compiler:

\begin{verbatim}
-I/damaris/install/prefix/include -I/path/to/dependencies/include
\end{verbatim}
If all dependencies and \Damaris{} have been installed in \installdir, for example, it becomes
\begin{verbatim}
-I$HOME/local/include
\end{verbatim}

Fortran programs may look for \texttt{damaris.mod}, which is also in the include directory
where \Damaris{} has been installed.

The linker will need the \Damaris{} library file \file{libdamaris.a} as well as other dependent libraries.
The following provides the necessary options for the linker:
\begin{verbatim}
-rdynamic
-L/damaris/install/prefix/lib -ldamaris -L/path/to/dependencies/lib \
-Wl,--whole-archive,-ldamaris,--no-whole-archive \
-lxerces-c \
-lboost_filesystem -lboost_system -lboost_date_time \
-lstdc++ -lrt -ldl
\end{verbatim}

The \texttt{-rdynamic} option asks the linker to add all symbols, not only used ones, 
to the dynamic symbol table. 
This option is needed for some uses of \texttt{dlopen} in \Damaris. It is possible that 
the option differs with some linkers.

\subsubsection{Running}

The deployment of a \Damaris-enabled simulation is that of a normal simulation:

\begin{verbatim}
	mpirun -np X -machinefile FILE ./sim
\end{verbatim}

Where X is the number of processes (in total), and FILE lists the hostnames on which to start the program.

\textbf{Note:} \Damaris{} doesn't use any a priori knowledge about rank-to-cores binding and is able
to start the proper number of servers at the right location even if process ranks are not consecutive 
in the nodes. It leverages the processor's name to group processes, or platform specific information 
such as the host's personality on BG/P. Yet, it is important that the same number of cores are used
in each node.

%==============================================================================%
%==============================================================================%
\section{Describing and writing data}

Now that we have a working simulation instrumented with \Damaris, it is time to write data.
This section explains how to describe data in the XML file and how to send it to \Damaris.
All the data items must be described in the XML configuration within the \texttt{<data>} section.

\subsection{Parameters}

Parameters are simple values associated with a name and a type. It allows to
easily change some important values of the simulation without changing all the variables
and objects that depend on them. Parameters can also be used as input of the simulation itself. 
Listing~\ref{parameterXML} shows how to define a parameter with a default value.
Parameters must be defined at the root of the \texttt{<data>}  section, not within nested groups
(later explained).

\noindent\begin{minipage}{\textwidth}
\vspace{0.5cm}
\lstset{language=XML,caption={Parameters descriptions},label=parameterXML}
\lstinputlisting[language=XML]{listings/parameter.xml}
\end{minipage}

Parameters must have a valid \texttt{name}, i.e., it should not contain any space or special symbols,
and should not start with a digit. Basically any name valid as variable name in C or Fortran is a valid
parameter name in \Damaris. The \texttt{type} attribute must be one of the types listed in 
Appendix~\ref{sec:types}. The provided \texttt{value} must be acceptable for the specified type. For instance
\texttt{value="xyz"} is not valid for an integer parameter.

Listing~\ref{parameterC} and~\ref{parameterF} show how to get and set parameters at run time
from the simulation. \textbf{Note:} when a process modifies a parameter, this modification is local and
not visible to other processes or to dedicated cores/nodes. The user is responsible for synchronizing all
processes when a parameter has to be changed globally.

\noindent\begin{minipage}{\textwidth}
\vspace{0.5cm}
\lstset{language=C,caption={Setting and getting parameters in C},label=parameterC}
\lstinputlisting[language=C]{listings/parameter.c}
\end{minipage}

\noindent\begin{minipage}{\textwidth}
\vspace{0.5cm}
\lstset{language=Fortran,caption={Setting and getting parameters in Fortran},label=parameterF}
\lstinputlisting[language=Fortran]{listings/parameter.f90}
\end{minipage}

Parameters are especially useful to define layouts, which are described in the next section.

\subsection{Layouts}

Layouts describe the shape of the data. \Damaris{} separates this description from the description
of the variables themselves since several variables can have the same layout.
A layout is characterized by a \texttt{name}, a base \texttt{type} and a list of \texttt{dimensions}.
Two examples are given in Listing~\ref{layoutXML}.

The \texttt{name} of a layout can be any string not including the ``/'' character 
(it can include special characters, white spaces, numbers, etc. even
though we advise to keep it simple and use classic C identifiers). 
The name cannot be the name of a basic type (such as \texttt{int}).
The \texttt{type} should be one of the basic types listed in Appendix~\ref{sec:types}.
Finally the list of dimensions is a comma-separated list of arithmetical expression featuring
+, -, *, /, \%, parenthesis, numbers and any defined int parameters (as of today, only int and integer 
parameters are allowed, short and long will produce an error).
In the following examples, the first layout has three dimensions, the second one has two dimensions.

When a layout depends on parameters, any modification of the parameters will automatically modify
the layout. Like parameters, the layout is only locally affected and the modification is not propagated to
other processes. Making a layout depend on a parameter and changing the parameter at run time 
can be very useful for particle-based simulations, where the number of particles is different at each process,
or simply to make the XML file independent of the simulation's size and avoid changing the parameters
every time the size changes.

The description of a layout will be used when writing a variable. Note that a modification in 
a layout at run time (through a modification of a parameter) will not affect previously written iterations of a
variable. It only affects the upcoming ones.

\noindent\begin{minipage}{\textwidth}
\vspace{0.5cm}
\lstset{language=XML,caption={Layouts descriptions},label=layoutXML}
\lstinputlisting[language=XML]{listings/layout.xml}
\end{minipage}

\textbf{Note:} \texttt{string} and \texttt{label} data types are variable-length types, thus
the dimension of the layout corresponds to the number of characters that the variable can store.

\subsection{Variables and groups}

\subsubsection{XML description}

Actual data is described through the \texttt{<variable>} and \texttt{<group>} nodes, which
allow to build a hierarchy of variables. An example is given in Listing~\ref{varXML}.
Each variable must be given a \texttt{name}, and be associated with a defined \texttt{layout}.
These are the two mandatory attributes. The \texttt{time-varying} attribute indicates whether
a variable is expected to be written at every iteration, or just once at the beginning of the
simulation (i.e., before the first call to \function{damaris\_end\_iteration}). 
The \texttt{visualizable} attribute indicates that a variable is visualizable
by visualization backends (for instance, coordinate arrays are not visualizable). 
In the following example, we expect the \texttt{x\_coordinates} variable
to be the coordinate of points in a rectilinear grid. This data is not itself visualizable, however
a rectilinear grid (later described in Chapter~\ref{chap:VisIt}) will be a visualizable object.
Note that one can use a basic type instead of a layout, as for the \texttt{simple var} variable: 
basic types are already interpreted as layouts.

\noindent\begin{minipage}{\textwidth}
\vspace{0.5cm}
\lstset{language=XML,caption={Variables and groups descriptions},label=varXML}
\lstinputlisting[language=XML]{listings/var.xml}
\end{minipage}

\subsubsection{Relative and absolute names}

Variables and layouts can be defined within groups. In Listing~\ref{varXML}, the relative name of the 
temperature variable is \emph{``temperature''}, while its absolute name is \emph{``my group/temperature''}.
The same goes for the name of layouts.

When \Damaris{} searches for a layout associated with a variable, it first looks inside
the group where the variable is defined, then in the parent group, and so on. It is thus possible to
refer to a layout either with its absolute name (if the layout is located in a different group) or
with a relative name if the layout can be found in the same group hierarchy.

\subsubsection{Writing full variables}

Now that the data is described, we can write it from the simulation. 
Listings~\ref{writingVarC} and~\ref{writingVarF90} present how to write a variable. 
The full name of the variable should be provided.

\noindent\begin{minipage}{\textwidth}
\vspace{0.5cm}
\lstset{language=C,caption={Writing a variable to \Damaris{} in C},label=writingVarC}
\lstinputlisting[language=C]{listings/write.c}
\end{minipage}

\noindent\begin{minipage}{\textwidth}
%\vspace{0.5cm}
\lstset{language=Fortran,caption={Writing a variable to \Damaris{} in Fortran},label=writingVarF90}
\lstinputlisting[language=Fortran]{listings/write.f90}
\end{minipage}

\subsubsection{Writing multiple domains}

By default, \Damaris{} expects one block of data per client and per variable.
It may be necessary for a client to write multiple blocks of a single variable. 
To do so, the number of domains have to be specified in the \texttt{domains} section of the 
XML file, as shown in Listing~\ref{writingBlocksXML}.

\noindent\begin{minipage}{\textwidth}
\vspace{0.5cm}
\lstset{language=XML,caption={Specifying multiple domains per client},label=writingBlocksXML}
\lstinputlisting[language=XML]{listings/blocks.xml}
\end{minipage}

This number is a maximum, a client may write less domains, but not more. Besides, if a client
writes $N$ domains, it must identify them from $0$ to $N-1$.
Listings~\ref{writingBlocksC} and~\ref{writingBlocksF90} show how to write multiple blocks.
Each block is expected to have the size and shape defined in the layout associated with the variable.

\noindent\begin{minipage}{\textwidth}
\vspace{0.5cm}
\lstset{language=C,caption={Writing a chunk of variable to \Damaris{} in C},label=writingBlocksC}
\lstinputlisting[language=C]{listings/blocks.c}
\end{minipage}

\noindent\begin{minipage}{\textwidth}
%\vspace{0.5cm}
\lstset{language=Fortran,caption={Writing a chunk of variable to \Damaris{} in Fortran},label=writingBlocksF90}
\lstinputlisting[language=Fortran]{listings/blocks.f90}
\end{minipage}

\subsubsection{Direct access to the shared memory}

The API presented above has the disadvantage of copying local data into the shared memory.
A more efficient way of proceeding consists of getting direct access to the shared memory.
Listings~\ref{allocateC} and~\ref{allocateF90} present this capability.

If \function{damaris\_alloc} fails, it will produce a NULL pointer. Otherwise, a valid pointer
to an allocated region of shared memory is produced.
After writing the data to the returned buffer, a call to \function{damaris\_commit} will notify the server
of the presence of new data. After this call, the user is not expected to write the data anymore.
Finally \function{damaris\_clear} indicates that the client delegates full responsibility to the
servers for the data. It is not supposed to be read nor written anymore.
Equivalent functions (\function{damaris\_alloc\_block} and \function{damaris\_alloc\_block\_f}) 
exist to allocate blocks when each client handles multiple domains.

\noindent\begin{minipage}{\textwidth}
\vspace{0.5cm}
\lstset{language=C,caption={Allocate a variable in shared memory from C},label=allocateC}
\lstinputlisting[language=C]{listings/alloc.c}
\end{minipage}

\noindent\begin{minipage}{\textwidth}
%\vspace{0.5cm}
\lstset{language=Fortran,caption={Allocate a variable in shared memory from Fortran},label=allocateF90}
\lstinputlisting[language=Fortran]{listings/alloc.f90}
\end{minipage}

Du to the C/Fortran interface, the use of \function{damaris\_alloc\_f} is more complex in Fortran than in C.
The \Damaris{} module should be used (this module is located where \Damaris{} has been installed). 
An additional call to \function{c\_f\_pointer} is mandatory to convert the returned C pointer
into a valid Fortran array. This function takes a shape array to provide the extents along each dimensions.

Note that it can be desirable,
in order to implement efficient double-buffering, that a client waits some iterations before actually committing 
a variable. Different versions of the function are available:
\begin{itemize}
\item \function{damaris\_commit(const char* var)} : \\ commits all the blocks of the current iteration;
\item \function{damaris\_commit\_block(const char* var, int32\_t block)} : \\ commits one specific block of the current iteration;
\item \function{damaris\_commit\_iteration(const char* var, int32\_t iteration)} : \\ commits all the blocks of a specific iteration;
\item \function{damaris\_commit\_block\_iteration(const char* var, int32\_t block, int32\_t iteration)} \\ commits one specific block of a specific iteration.
\end{itemize}

Equivalent functions are available in Fortran: \function{damaris\_commit\_f},
\function{damaris\_commit\_block\_f},\\ \function{damaris\_commit\_iteration\_f} and \function{damaris\_commit\_block\_iteration\_f}.
All take an \emph{ierr} integer in addition to the same parameters as the C functions.

\subsubsection{Error handling}\label{sec:error}

Since \function{damaris\_write} and \function{damaris\_alloc} (and corresponding functions for writing or
allocating blocks) both require to allocate a portion of shared memory, a call may fail if the shared memory
is full. When this happens \emph{all subsequent calls to \function{damaris\_write} or \function{damaris\_alloc}
up to the next call to \function{damaris\_end\_iteration} will return with an error without attempting
to allocate memory}. A special signal will be sent to all the servers when
calling \function{damaris\_end\_iteration}, which informs the servers that some data are missing
for this iteration. By default, the dedicated cores will not update potentially connected visualization backends,
and will delete from memory the data that has been written successfully for this iteration.
Overwriting this default behavior will be covered in the next chapter.



%%%%%%%%%%%%%%%%%%%%%%%%%%%%%%%%%%%%%%%%%%%%%%%%%%%%
%%%%%%%%%%%%%%%%%%%%%%%%%%%%%%%%%%%%%%%%%%%%%%%%%%%%
%%%%%%%%%%%%%%%%%%%%%%%%%%%%%%%%%%%%%%%%%%%%%%%%%%%%
%%%%%%%%%%%%%%%%%%%%%%%%%%%%%%%%%%%%%%%%%%%%%%%%%%%%
%%%%%%%%%%%%%%%%%%%%%%%%%%%%%%%%%%%%%%%%%%%%%%%%%%%%
%%%%%%%%%%%%%%%%%%%%%%%%%%%%%%%%%%%%%%%%%%%%%%%%%%%%
\chapter{Writing and Calling Plugins}\label{Plugins}

Writing data to \Damaris{} is cool, but what should the server do with that? For now the shared memory
buffer will simply get quickly filled and future write operations will fail. The role of processing data
(and removing them from the shared memory) is delegated to plugins, which are triggered by events
sent by the simulation to the servers.
This section explains how to program a plugin for \Damaris{} and how to trigger it from the simulation.

%==============================================================================%
%==============================================================================%
\section{Describing actions}

The interactions between the simulation processes and the dedicated cores are based on events.
These events are sent by simulation processes to the servers.
Like other \Damaris{} objects, these events have to be described in the XML configuration file,
more specifically within the \texttt{<actions>} section.
Listing~\ref{actionXML} presents the description of such an \texttt{event}.
An \texttt{event} is described by a unique \texttt{name}, and an \texttt{action}, which corresponds to
the name of a C++ function. If this function is located in a shared \texttt{library} (.so file), the name
of the library has to be provided. \Damaris{} will look for shared libraries within the directories
specified by the LD\_LIBRARY\_PATH environment variable. If no library is specified, \Damaris{}
will look for the function within its own binary code.
Finally one important characteristic of events is the \texttt{scope}:
\begin{itemize}
	\item \textbf{``core''} indicates that every time a simulation process sends the event, the
	corresponding action is triggered by the the server receiving the event. If 15 clients 
	in a 16-cores node send the same event to their server running on a dedicated core, 
	the corresponding action is triggered 16 times within the server.
	\item \textbf{``group''} indicates that a server has to wait for all its clients to have sent
	the same event (at the same iteration) before triggering the action. If the 15 clients
	send the event, the action is triggered only once. If one of the processes does not send the
	event, the corresponding action is never triggered.
	\item \textbf{``bcast''} indicates that if one client sends this signal, then all the servers 
	will perform the corresponding action, not only the server responsible for this
	client. MPI communications can safely be used within an action triggered by such an event.
	It is thus the right type of event to use if the plugin is collective across servers.
\end{itemize}

\noindent\begin{minipage}{\textwidth}
\vspace{0.5cm}
\lstset{language=XML,caption={Actions description},label=actionXML}
\lstinputlisting[language=XML]{listings/action.xml}
\end{minipage}

The semantics of these scopes with respect to the order of triggered actions is the following.
\begin{itemize}
\item Two ``core'' events sent by the same process will be received in the same order by the dedicated core.
\item If an event (whatever the scope) is sent after writing or committing a variable, it is ensured that the
server has received the variable before triggering the event's action.
\item Two ``group'' events sent by all the clients of a group in the same order will trigger their actions in
the same order in the server. No particular order should be assumed if different processes send the
same event in a different order.
\item Two ``bcast'' events sent by the same client will trigger their actions
in the same order in all the servers. No particular order should be assumed if different processes send
the a ``bcast'' event.
\item If a client sends two events $e_1$ and $e_2$, if $e_1$ has a ``core'' scope 
then it is ensured that its corresponding action will be executed before the action of 
$e_2$. 
\item However, if $e_1$ has another scope than ``core'', no ordering can be assumed.
\item To force an ordered execution of $e_1$ and $e_2$ where the scope of $e_1$ is not ``core'',
global synchronization barriers can be used in the simulation before sending $e_2$.
\end{itemize}

%==============================================================================%
%==============================================================================%
\section{\emph{Hello World!} plugin example}

Now let's take a look at a C++ function that can be called by \Damaris. An example of such a function
is provided in Listing~\ref{PluginCPP}. Note the use of \texttt{extern "C"} to prevent C++ name mangling.

\noindent\begin{minipage}{\textwidth}
\vspace{0.5cm}
\lstset{language=C++,caption={Example \Damaris{} plugin},label=PluginCPP}
\lstinputlisting[language=C++]{listings/plugin.cpp}
\end{minipage}

The first argument of such a function is a string representing the name of the event that triggered
the action (since several events can be connected to the same action). The second parameter
corresponds to the iteration at which the event has been sent. The third parameter is the rank
of the client that sent the event. For ``group'' actions, this source is set to $-1$.
Finally the last argument is only used by external plugins to send their own commands.

This first example only helps you start with plugins, more consideration on how to
access the data written by clients will be provided in Section~\ref{sec:internalAPI}.

%==============================================================================%
%==============================================================================%
\section{Compiling and linking C/C++ plugins}

To compile a plugin within a shared library, 
you need to create an object file from your source file with the -fPIC option,
then create a shared library:
\begin{verbatim}
g++ -fPIC -c something.cpp -I/path/to/damaris/include -I/path/to/damaris/dependencies/include
gcc -shared -Wl,-soname,libsomething.so -o libsomething.so something.o
\end{verbatim}

If you want to integrate your plugin in the application's code (for instance to avoid
loading the .so file from many processes at the same time, which could result in performance
degradation at large scale), simply compile the source of your plugin with your application. You must
dynamically link your simulation when compiling it.

%==============================================================================%
%==============================================================================%

\section{Sending events from the simulation}

To send events from the simulation, use the \function{damaris\_signal} function (in C) or \function{damaris\_signal\_f}
(in Fortran). These functions take the name of an event as parameter; this name should correspond to
an event described in the configuration file.

\noindent\begin{minipage}{\textwidth}
\vspace{0.5cm}
\lstset{language=C,caption={Sending events from a C program},label=EventC}
\lstinputlisting[language=C]{listings/event.c}
\end{minipage}

\noindent\begin{minipage}{\textwidth}
%\vspace{0.5cm}
\lstset{language=fortran,caption={Sending events from a Fortran program},label=EventF90}
\lstinputlisting[language=fortran]{listings/event.f90}
\end{minipage}

The prototype of these two functions are summarized in Table~\ref{tab:signalFunctions}. The returned
value (or the \emph{ierr} value in Fortran) is 0 in case of success, -1 in case of failure when sending the
event through the shared memory, and -2 if the name does not correspond to a defined event.

\begin{table}[h]
\centering
\begin{tabular}{|l|}
	\hline
   \textbf{C function} \\
   \hline
   \hline
   \function{int damaris\_signal(const char* event)}  \\
   \hline
   \hline
   \textbf{Fortran function} \\
   \hline
   \hline
   \function{damaris\_signal\_f(character* event, integer ierr)} \\
   \hline
\end{tabular}\caption{\Damaris{} signal functions}\label{tab:signalFunctions}
\end{table}

\section{Binding simulation's functions to events}

On some platforms, the \texttt{dlopen} function does not properly work, which prevents \Damaris{} from
being able to retrieve the plugin. In this case, another solution consists of providing the function's pointer
at run time.

To do so, \emph{all} the processes must call \texttt{damaris\_bind\_function(``event name'',function\_ptr)} \emph{before}
calling \texttt{damaris\_start} (so that both clients and servers are aware of the event), 
and in the \emph{same order} if multiple events are defined this way. 
The first argument to this function is the name of the event to define,
the second one is the pointer to a function that has the following signature, defined in Damaris.h:
\begin{verbatim}
typedef (*signal_t)(const char*,int32_t,int32_t,const char*);	
\end{verbatim}

This method is experimental, currently only available in C, and allows only a ``core'' scope. 
Besides, the event \emph{should not} be already defined in the configuration file.

%==============================================================================%
%==============================================================================%
\section{The internal \Damaris{} API}\label{sec:internalAPI}

In order to access the data from within a plugin, you will have to used \Damaris' data structures.
The main structure is the \texttt{VariableManager}, which can be used to access \texttt{Variable} instances
and, from there, \texttt{Block} instances (which correspond to our notion of ``domains''), and \texttt{DataSpace}
instances, which hold data.
We invite the reader to refer to the Doxygen documentation of
\Damaris, which we hope is complete enough to understand how to retrieve data,
and to post requests on the \Damaris{} mailing list if you need any help.

\vspace{0.5cm}

\noindent\textbf{Note:} To use the internal structures and objects of \Damaris, it is necessary
to include the proper header files. These header files have been installed in a \texttt{damaris} folder
in the \texttt{include} folder where \Damaris{} was installed.

\section{Error handling through events}

If the simulation did not manage to write all its data for a
given iteration, a special signal is send to all the servers to ask them not to update
visualization backends and to erase all remaining data for this iteration and all past iterations.
This behavior can be overwritten by attaching an error handler to an event,
as exemplified in the following XML codes. The event \emph{must} have a \textbf{``core''} scope,
and the event or script must be defined somewhere in the configuration file.

\noindent\begin{minipage}{\textwidth}
\vspace{0.5cm}
\lstset{language=XML,caption={XML information to attach an event to errors.},label=errorXML}
\lstinputlisting[language=XML]{listings/errorevent.xml}
\end{minipage}



%==============================================================================%
%==============================================================================%
\chapter{Dedicated Cores, Dedicated Nodes and Synchronous Mode}\label{modes}

This chapter presents the different modes that \Damaris{} can use: dedicated cores, dedicated nodes
and synchronous mode.

\section{Enabling different modes}

Different modes can be enabled simply by setting the number of dedicated cores or dedicated nodes
in the XML file, as exemplified in Listing~\ref{synchronousXML}.

\noindent\begin{minipage}{\textwidth}
\vspace{0.5cm}
\lstset{language=XML,caption={Setting the number of dedicated cores or dedicated nodes.},label=synchronousXML}
\lstinputlisting[language=XML]{listings/synch.xml}
\end{minipage}

\section{Synchronous mode}

The synchronous mode, or ``time-partitioning'', is used when cores=``0'' and nodes=``0''. In this configuration,
all actions will be executed by the clients themselves. In other words, a client is its own server.
Their is thus no distinction between the ``core'' and the ``group'' scope, and \textbf{``bcast'' events
are not executed} (because their is no application logic to asynchronously receive the event
in other clients, and they can be replaced by simple MPI\_Bcast logics).

\section{Dedicated cores}

When the \emph{cores} field has a value $N > 0$, dedicated cores are used. This number of dedicated
cores must divide the number of cores in a node. When $N=1$, all clients in a node are connected to
a server running on a dedicated core of the same node. When $N > 1$, the client cores in a node
are gathered into $N$ groups, and each group is connected to a server on a dedicated core.
Note that whatever the number of servers, only one shared-memory buffer is created, with a size
specified in the XML file.

\section{Dedicated nodes}

When the \emph{cores} field is set to 0 or unspecified, and the \emph{nodes} field is set to $N > 0$,
then $N$ nodes will be dedicated to run servers. $N$ must divide the total number of nodes of the
application. On each of the dedicated nodes, \Damaris{} will spawn as many
server instances as cores. Each dedicated node holds a single buffer of the specified size for all
its servers. The clients will be gathered in equally-sized groups and each group will be connected to
a server.

%%%%%%%%%%%%%%%%%%%%%%%%%%%%%%%%%%%%%%%%%%%%%%%%%%%%
%%%%%%%%%%%%%%%%%%%%%%%%%%%%%%%%%%%%%%%%%%%%%%%%%%%%
%%%%%%%%%%%%%%%%%%%%%%%%%%%%%%%%%%%%%%%%%%%%%%%%%%%%
%%%%%%%%%%%%%%%%%%%%%%%%%%%%%%%%%%%%%%%%%%%%%%%%%%%%
%%%%%%%%%%%%%%%%%%%%%%%%%%%%%%%%%%%%%%%%%%%%%%%%%%%%
%%%%%%%%%%%%%%%%%%%%%%%%%%%%%%%%%%%%%%%%%%%%%%%%%%%%
%\chapter{Python Interface}\label{Python}

%\Damaris{} allows plugins written in Python, as soon as the Python system has been
enabled when compiling \Damaris{}. This section explains how to install and use the Python
plugins system.

%==============================================================================%
%==============================================================================%
\section{Installation requirements}

Download Python (\Damaris{} has been tested with version 2.6.8 so far, even though any 2.x version
should work. Because of compatibility issues with Boost, it is not ensured that version 3.x will work) at 
\url{http://www.python.org/download/releases/}.
Untar and install using
\begin{verbatim}
./configure --prefix=$HOME/local --disable-shared
make
make install
\end{verbatim}

This will install Python in your \installdir directory. \textbf{Important note:} if you have installed VisIt,
VisIt provides a local python installation. In order to avoid conflicts when using both Python and VisIt,
use the one provided by VisIt (located in \texttt{/path/to/visit/python}).

\Damaris{} also needs NumPy, from \url{http://www.scipy.org/Download/}. Download the source archive,
then untar it install it:

\begin{verbatim}
python setup.py install
\end{verbatim}
\textbf{Note:} make sure the \textbf{python} command you runs calls the Python interpreter you just installed
or the one from VisIt, and NOT the default interpreter of your platform.

Now that Python and NumPy are both installed, recompile \Damaris{} (make sure to change
the lines in the CMakeLists.txt that include PYTHON\_ROOT and NUMPY\_INCLUDE\_DIR).

\section{XML configuration}

In the XML configuration of your simulation, you can provide some information related to
Python, by writing the lines in Listing~\ref{PythonXML} after the \texttt{</actions>} tag.

\noindent\begin{minipage}{\textwidth}
\vspace{0.5cm}
\lstset{language=XML,caption={Providing Python options},label=PythonXML}
\lstinputlisting[language=XML]{listings/pythonpath.xml}
\end{minipage}

The \texttt{<path>} section corresponds to the PYTHON\_PATH environment variable,
while the \texttt{<home>} section corresponds to the PYTHON\_HOME environment variable.
These options are useful as it may be difficult on some machines to properly forward environment
variables to compute nodes.

%==============================================================================%
%==============================================================================%
\section{Python scripts as plugins}

Python plugins in \Damaris{} are written in Python scripts. To connect an event to
a script, follow the example in Listing~\ref{PluginPythonXML}. By calling \function{DC\_signal}
with the name provided in the configuration, the corresponding script will be executed.
\Damaris{} currently support only Python scripts, thus the \texttt{language} tag will always contain
"python". The \texttt{scope} tat has the same semantic than for events.

\noindent\begin{minipage}{\textwidth}
\vspace{0.5cm}
\lstset{language=XML,caption={Describing a Python script},label=PluginPythonXML}
\lstinputlisting[language=XML]{listings/python.xml}
\end{minipage}

%==============================================================================%
%==============================================================================%
\section{Accessing \Damaris{} data from Python}

In the Python scripts, everything related to \Damaris{} is located in the \texttt{damaris} module. 
Use \texttt{import damaris} and \texttt{import numpy} to access these functionalities.
Table~\ref{tab:pythonAPI} presents how to access chunks of data from Python.

\begin{table}[h]
\centering{}
   \begin{tabular}{|l|l|l|}
       \hline
       Statement & Description \\
       \hline
       \hline
       damaris.iteration & Iteration at which the event has been sent. \\
       damaris.source & Rank of the process that sent the event.\\
       damaris.clear & Remove all the chunks from all the variables. \\
       var = damaris.open("group/varname") & Opens a variable.\\
       var.name & Name of the variable.\\
       var.fullname & Full name (including groups) of the variable.\\
       var.unit & Unit of the variable.\\
       var.description & Description from the XML file. \\
       layout = var.layout & Layout of the variable.\\
       list = var.select(\{"iteration": x, "source": y\}) & Select a list of chunks by source and/or iteration.\\
       var.remove(c) & Removes a Chunk (free the memory). \\
       var.clear() & Removes all the chunks attached to the variable. \\
       layout.name & Name of the layout. \\
       layout.type & Type of the layout (string). \\
       layout.extents & Array of dimensions. \\
       chunk.source & Source of the Chunk (process rank). \\
       chunk.iteration & Iteration of the Chunk. \\
       chunk.type & Type of the chunk (string). \\
       chunk.lower\_bounds & List of lower bounds along each dimension. \\
       chunk.upper\_bounds & List of upper bounds along each dimension. \\
       chunk.data & NumPy array of the chunk's data. \\
       \hline
   \end{tabular}\caption{\Damaris{} Python API}\label{tab:pythonAPI}
\end{table}


%%%%%%%%%%%%%%%%%%%%%%%%%%%%%%%%%%%%%%%%%%%%%%%%%%%%
%%%%%%%%%%%%%%%%%%%%%%%%%%%%%%%%%%%%%%%%%%%%%%%%%%%%
%%%%%%%%%%%%%%%%%%%%%%%%%%%%%%%%%%%%%%%%%%%%%%%%%%%%
%%%%%%%%%%%%%%%%%%%%%%%%%%%%%%%%%%%%%%%%%%%%%%%%%%%%
%%%%%%%%%%%%%%%%%%%%%%%%%%%%%%%%%%%%%%%%%%%%%%%%%%%%
%%%%%%%%%%%%%%%%%%%%%%%%%%%%%%%%%%%%%%%%%%%%%%%%%%%%
\chapter{VisIt interface}\label{chap:VisIt}

\Damaris{} can be used to perform in situ visualization with VisIt through the \emph{libsimV2} interface. 
VisIt must be installed on your platform (see the VisIt website at \url{https://wci.llnl.gov/codes/visit/}).
Follow the installation process of \Damaris{} presented in Chapter~\ref{chap:downloadingAndInstalling}
and enable VisIt by providing the VISIT\_ROOT 
VisIt install directory in the CMakeLists.txt (see Table~\ref{tab:cmake}).
\textbf{Warning:} \emph{libsimV2} seems to have some problems on some platforms. We successfully
tested \Damaris{} with VisIt 2.5.2 and 2.8.0. Versions earlier than 2.5.2 will not work with \Damaris{},
and du to various bugs in \emph{libsimV2}, we haven't managed to have \Damaris{} use any of the
versions between 2.5.2 and 2.8.0. Version 2.8.0 seems to still suffer from problems on some platforms.

%==============================================================================%
%==============================================================================%
\section{VisIt options}

VisIt has to be enabled through the configuration file as shown in
Listing~\ref{VisItXML}. The \texttt{<path>} tag locates the installation
directory of VisIt, it is used by the simulation to load VisIt plugins.
The \texttt{<options>} tag provides an equivalent to VisIt's command line options (see
the VisIt manual). Most of the time, the \texttt{<options>} section is not required.

\noindent\begin{minipage}{\textwidth}
\vspace{0.5cm}
\lstset{language=XML,caption={Providing VisIt options},label=VisItXML}
\lstinputlisting[language=XML]{listings/visit.xml}
\end{minipage}

%==============================================================================%
%==============================================================================%
\section{Description of meshes, curves and fields}

\subsection{Meshes}

The current version of \Damaris{} supports the following meshes:
rectilinear meshes, curvilinear meshes, and point meshes. This section shows how to expose
a mesh and how to map a variable onto a mesh.
Just like other \Damaris{} objects, meshes have to be described in the XML file, as shown
in Listing~\ref{MeshVisItXML}. This kind of description should be placed in the \texttt{<data>} section
of the configuration, at the root or within any subgroup.

\noindent\begin{minipage}{\textwidth}
\vspace{0.5cm}
\lstset{language=XML,caption={Describing a Mesh in \Damaris{}},label=MeshVisItXML}
\lstinputlisting[language=XML]{listings/mesh.xml}
\end{minipage}

A mesh is characterized by a \texttt{name}, a \texttt{type} (currently only ``rectilinear'', ``curvilinear'' or
``point'' are allowed), and a topological dimension (for instance a surface in a 3D space is characterized by three
coordinates for each of its points, but has a 2D topology).

Finally two to three coordinates have to be provided through the \texttt{<coord>} tag. 
These coordinates refer to existing variables from the configuration.
The name of these variables can be absolute or relative, as it is search the same way a variable searches
its layout. The \texttt{unit} of a coordinate, when provided, hides the \texttt{unit} of the corresponding variable.
The \texttt{label} is used when drawing figures.

\subsection{Fields}

A field is a value that can be placed on each node or each zone of a mesh.
To allow a described variable to be mapped onto a mesh, some new attributes have to be added in its description.
Listing~\ref{xml:field} shows the description of a variable with attributes related to visualization.

\noindent\begin{minipage}{\textwidth}
\vspace{0.5cm}
\lstset{language=XML,caption={Extended variable description to expose it as a field},label=xml:field}
\lstinputlisting[language=XML]{listings/field.xml}
\end{minipage}

The most important attribute is \texttt{mesh}, which is a reference to an existing mesh on which
to draw the field. This reference should be an absolute name (\Damaris{} is not yet capable of searching
for a local mesh within the group where the variable is defined). 
The \texttt{centering} attribute can be either ``nodal'' (values correspond to vertices 
of the mesh) or ``zonal'' (values correspond to zones in the mesh). The \texttt{type} should be
``scalar'' (other types are not yet supported by \Damaris).

By default, \texttt{visualizable} and \texttt{time-varying} are already set to ``true'', \texttt{unit} can be omitted.
The other attributes (\texttt{type} and \texttt{centering}) can be omitted if the variable is not concerned by
visualization.
The ``rule'' is that variables that serve as coordinates in meshes are not visualizable, while
other variable are considered as fields and are visualizable, thus must have an attached mesh.

\subsection{Curves}

\Damaris{} can expose curves the same way it exposes meshes. Listing~\ref{CurveVisItXML}
shows how to expose a curve. A curve has exactly two coordinates (the $x$ and $y$ coordinates
of the points of the curve) refering to variables the same way meshes use coordinate variables.

An important difference between meshes and curves is that a curve is only produced by
the client 0 of the simulation. That is, only this client has to write the coordinate variables.
If other clients write coordinate variables, they will be discarded by VisIt.

\noindent\begin{minipage}{\textwidth}
\vspace{0.5cm}
\lstset{language=XML,caption={Describing a Mesh in \Damaris{}},label=CurveVisItXML}
\lstinputlisting[language=XML]{listings/curve.xml}
\end{minipage}

%==============================================================================%
%==============================================================================%
\section{External actions and interactivity}

Events can be set as ``external'' by adding the \texttt{external="true"} attribute to their
description. An external event can be fired from the VisIt client itself. These events are exposed as commands
and are accessible through the VisIt viewer's commands interface.
Clicking on a command will trigger the corresponding event in all dedicated processes (or all processes
in synchronous mode). It is ensured that all the actions will be triggered at the same time, thus
the actions can safely use global communications using MPI.
Triggering an event from VisIt is equivalent to sending a ``bcast''-type event from a client.

%%%%%%%%%%%%%%%%%%%%%%%%%%%%%%%%%%%%%%%%%%%%%%%%%%%%
%%%%%%%%%%%%%%%%%%%%%%%%%%%%%%%%%%%%%%%%%%%%%%%%%%%%
%%%%%%%%%%%%%%%%%%%%%%%%%%%%%%%%%%%%%%%%%%%%%%%%%%%%
%%%%%%%%%%%%%%%%%%%%%%%%%%%%%%%%%%%%%%%%%%%%%%%%%%%%
%%%%%%%%%%%%%%%%%%%%%%%%%%%%%%%%%%%%%%%%%%%%%%%%%%%%
%%%%%%%%%%%%%%%%%%%%%%%%%%%%%%%%%%%%%%%%%%%%%%%%%%%%

\appendix
\chapter{\Damaris{} basic types}\label{sec:types}

\begin{table}[h]
\centering
   \begin{tabular}{|c|c|c|l|}
      \hline
      \Damaris{} type & C type & Fortran type & Description \\
      \hline
short & short int & integer*2 &
                    2 bytes integer value.\\
int & int & integer &
                    4 bytes integer value.\\
integer & int & integer &
                    4 bytes integer value.\\
long & long int & integer*8 &
                    8 bytes integer value.\\
float & float & real*4 &
                    4 bytes float value.\\
real & float & real*4 &
                    4 bytes float value.\\
double & double & real*8 &
                    8 bytes float value.\\
char & char & character &
                    1 byte character value.\\
character & char & character &
                    1 byte character value.\\
string & char[] & character(:) &
                    Variable length null-terminated string.\\
label & ? & ? &
                    Not implemented yet.\\
                    \hline
   \end{tabular}
   \caption{Correspondence between \Damaris{} types, C and Fortran types.}
\end{table}

\printindex

\end{document}
