This chapter presents the different modes that \Damaris{} can use: dedicated cores, dedicated nodes
and synchronous mode.

\section{Enabling different modes}

Different modes can be enabled simply by setting the number of dedicated cores or dedicated nodes
in the XML file, as exemplified in Listing~\ref{synchronousXML}.

\noindent\begin{minipage}{\textwidth}
\vspace{0.5cm}
\lstset{language=XML,caption={Setting the number of dedicated cores or dedicated nodes.},label=synchronousXML}
\lstinputlisting[language=XML]{listings/synch.xml}
\end{minipage}

\section{Synchronous mode}

The synchronous mode, or ``time-partitioning'', is used when cores=``0'' and nodes=``0''. In this configuration,
all actions will be executed by the clients themselves. In other words, a client is its own server.
Their is thus no distinction between the ``core'' and the ``group'' scope, and \textbf{``bcast'' events
are not executed} (because their is no application logic to asynchronously receive the event
in other clients, and they can be replaced by simple MPI\_Bcast logics).

\section{Dedicated cores}

When the \emph{cores} field has a value $N > 0$, dedicated cores are used. This number of dedicated
cores must divide the number of cores in a node. When $N=1$, all clients in a node are connected to
a server running on a dedicated core of the same node. When $N > 1$, the client cores in a node
are gathered into $N$ groups, and each group is connected to a server on a dedicated core.
Note that whatever the number of servers, only one shared-memory buffer is created, with a size
specified in the XML file.

\section{Dedicated nodes}

When the \emph{cores} field is set to 0 or unspecified, and the \emph{nodes} field is set to $N > 0$,
then $N$ nodes will be dedicated to run servers. $N$ must divide the total number of nodes of the
application. On each of the dedicated nodes, \Damaris{} will spawn as many
server instances as cores. Each dedicated node holds a single buffer of the specified size for all
its servers. The clients will be gathered in equally-sized groups and each group will be connected to
a server.