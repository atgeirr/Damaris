\Damaris{} allows plugins written in Python, as soon as the Python system has been
enabled when compiling \Damaris{}. This section explains how to install and use the Python
plugins system.

%==============================================================================%
%==============================================================================%
\section{Installation requirements}

Download Python (\Damaris{} has been tested with version 2.6.8 so far, even though any 2.x version
should work. Because of compatibility issues with Boost, it is not ensured that version 3.x will work) at 
\url{http://www.python.org/download/releases/}.
Untar and install using
\begin{verbatim}
./configure --prefix=$HOME/local --disable-shared
make
make install
\end{verbatim}

This will install Python in your \installdir directory. \textbf{Important note:} if you have installed VisIt,
VisIt provides a local python installation. In order to avoid conflicts when using both Python and VisIt,
use the one provided by VisIt (located in \texttt{/path/to/visit/python}).

\Damaris{} also needs NumPy, from \url{http://www.scipy.org/Download/}. Download the source archive,
then untar it install it:

\begin{verbatim}
python setup.py install
\end{verbatim}
\textbf{Note:} make sure the \textbf{python} command you runs calls the Python interpreter you just installed
or the one from VisIt, and NOT the default interpreter of your platform.

Now that Python and NumPy are both installed, recompile \Damaris{} (make sure to change
the lines in the CMakeLists.txt that include PYTHON\_ROOT and NUMPY\_INCLUDE\_DIR).

\section{XML configuration}

In the XML configuration of your simulation, you can provide some information related to
Python, by writing the lines in Listing~\ref{PythonXML} after the \texttt{</actions>} tag.

\noindent\begin{minipage}{\textwidth}
\vspace{0.5cm}
\lstset{language=XML,caption={Providing Python options},label=PythonXML}
\lstinputlisting[language=XML]{listings/pythonpath.xml}
\end{minipage}

The \texttt{<path>} section corresponds to the PYTHON\_PATH environment variable,
while the \texttt{<home>} section corresponds to the PYTHON\_HOME environment variable.
These options are useful as it may be difficult on some machines to properly forward environment
variables to compute nodes.

%==============================================================================%
%==============================================================================%
\section{Python scripts as plugins}

Python plugins in \Damaris{} are written in Python scripts. To connect an event to
a script, follow the example in Listing~\ref{PluginPythonXML}. By calling \function{DC\_signal}
with the name provided in the configuration, the corresponding script will be executed.
\Damaris{} currently support only Python scripts, thus the \texttt{language} tag will always contain
"python". The \texttt{scope} tat has the same semantic than for events.

\noindent\begin{minipage}{\textwidth}
\vspace{0.5cm}
\lstset{language=XML,caption={Describing a Python script},label=PluginPythonXML}
\lstinputlisting[language=XML]{listings/python.xml}
\end{minipage}

%==============================================================================%
%==============================================================================%
\section{Accessing \Damaris{} data from Python}

In the Python scripts, everything related to \Damaris{} is located in the \texttt{damaris} module. 
Use \texttt{import damaris} and \texttt{import numpy} to access these functionalities.
Table~\ref{tab:pythonAPI} presents how to access chunks of data from Python.

\begin{table}[h]
\centering{}
   \begin{tabular}{|l|l|l|}
       \hline
       Statement & Description \\
       \hline
       \hline
       damaris.iteration & Iteration at which the event has been sent. \\
       damaris.source & Rank of the process that sent the event.\\
       damaris.clear & Remove all the chunks from all the variables. \\
       var = damaris.open("group/varname") & Opens a variable.\\
       var.name & Name of the variable.\\
       var.fullname & Full name (including groups) of the variable.\\
       var.unit & Unit of the variable.\\
       var.description & Description from the XML file. \\
       layout = var.layout & Layout of the variable.\\
       list = var.select(\{"iteration": x, "source": y\}) & Select a list of chunks by source and/or iteration.\\
       var.remove(c) & Removes a Chunk (free the memory). \\
       var.clear() & Removes all the chunks attached to the variable. \\
       layout.name & Name of the layout. \\
       layout.type & Type of the layout (string). \\
       layout.extents & Array of dimensions. \\
       chunk.source & Source of the Chunk (process rank). \\
       chunk.iteration & Iteration of the Chunk. \\
       chunk.type & Type of the chunk (string). \\
       chunk.lower\_bounds & List of lower bounds along each dimension. \\
       chunk.upper\_bounds & List of upper bounds along each dimension. \\
       chunk.data & NumPy array of the chunk's data. \\
       \hline
   \end{tabular}\caption{\Damaris{} Python API}\label{tab:pythonAPI}
\end{table}